This chapter includes our C++ code for numerical calculations of points of superstability for the logistic and skewed logistic maps.

The code was written in C++, compiled with g++ (GCC) 14.2.1 20250128, using GNU MP library version 6.3.0, and is shown below. 

It is also available at \url{https://github.com/quantifying-chaos/graph_qc/blob/main/cpp/numerical_feigenbaum_constants/main.cpp}.

% including the code 
\begin{lstlisting}[style=cppstyle]
// compile with -lgmpxx -lgmp
#include <algorithm>
#include <bits/stdc++.h>
#include <chrono>
#include <cmath>
#include <gmp.h>
#include <gmpxx.h>
#include <iomanip>
#include <iostream>
#include <stdexcept>
#include <vector>

#define PRECISION 4096 // in bits

using namespace std;

typedef void (*iter_map)(mpf_class &, mpf_class, mpf_class);
typedef mpf_class (*mpf_func)(mpf_class);

const mpf_class ALPHA = mpf_class(2.503, PRECISION);
const mpf_class DELTA = mpf_class(4.669, PRECISION);

template <typename T> void display(vector<T> &dis) {
  for (auto i : dis) {
    cout << i << endl;
  }
  cout << "length: " << dis.size() << endl;
}

mpf_class random_mpz(long low, long high) {
  if (high <= low) {
    throw std::invalid_argument(
        "high must be greater than low in random_mpz!");
  }
  static gmp_randclass r1(gmp_randinit_default);
  static int init = 0;
  if (!init) {
    r1.seed(std::chrono::system_clock::now()
                .time_since_epoch()
                .count());
    init = 1;
  }

  return low + r1.get_f() * (high - low);
}

void logistic(mpf_class &res, mpf_class lambda,
              mpf_class x) {
  res = 4 * lambda * x * (1 - x);
}

void logistic_skewed(mpf_class &res, mpf_class lambda,
                     mpf_class x) {
  res = lambda * x * (1 - x) * (1 - x);
}

// Iterate the functions n times, with input x and parameter
// lambda, and set the res to the result. the input map must
// have signature like logistic that is, void map(mpf_class
// &res, mpf_class lambda, mpf_class x)
void iterate(mpf_class &res, iter_map map, unsigned int n,
             mpf_class lambda, mpf_class x) {
  res = x;
  for (unsigned int i = 0; i < n; i++) {
    map(res, lambda, res);
  }
}

bool is_super_stable(
    iter_map map, mpf_class (*x_bar)(mpf_class),
    mpf_class lambda,
    unsigned int
        orbit_2_power, // at lambda, map has a stable
                       // 2^orbit_2_power orbit. This means
                       // we iterate the map
                       // 2^{orbit_2_power} times
    mpf_class threshold, unsigned int extra_iter) {
  mpf_class x = x_bar(lambda);
  mpf_class expected = x_bar(lambda);

  iterate(x, map, std::pow(2, orbit_2_power + extra_iter),
          lambda, x);
  if (abs(x - expected) < threshold) {
    return true;
  }
  return false;
}

mpf_class poke_lambda(
    iter_map map, mpf_class (*x_bar)(mpf_class),
    mpf_class lambda,
    unsigned int
        orbit_2_power, // at lambda, map has a stable
                       // 2^orbit_2_power orbit. This means
                       // we iterate the map
                       // 2^{orbit_2_power} times
    mpf_class poke_delta, unsigned int poke_times) {
  const mpf_class expected = x_bar(lambda);

  // create a list of finer lambda iterate 2**orbit_2_power
  // times. record their difference in abs in diff vector,
  // and pick the lambda corresponds to the smallest one
  std::vector<mpf_class> lambda_list;
  for (int i = 0; i < poke_times; i++) {
    lambda_list.push_back(lambda - poke_delta * i);
  }
  for (int i = 1; i < poke_times; i++) {
    lambda_list.push_back(lambda + poke_delta * i);
  }

  std::vector<mpf_class> diff;
  for (mpf_class val : lambda_list) {
    mpf_class x = expected;
    iterate(x, map, std::pow(2, orbit_2_power), val, x);
    diff.push_back(abs(x - expected));
  }

  auto it =
      std::min_element(std::begin(diff), std::end(diff));
  return lambda_list[std::distance(std::begin(diff), it)];
}

mpf_class poke_lambda_iterate(
    iter_map map, mpf_class (*x_bar)(mpf_class),
    mpf_class lambda, // poke the lambda value around this
    unsigned int
        orbit_2_power, // at lambda, map has a stable
                       // 2^orbit_2_power orbit. This means
                       // we iterate the map
                       // 2^{orbit_2_power} times
    mpf_class poke_delta, unsigned int poke_times,
    unsigned int iter_n) {
  for (unsigned int i = 0; i < iter_n; i++) {
    lambda = poke_lambda(map, x_bar, lambda, orbit_2_power,
                         poke_delta, poke_times);
    poke_delta /= 2;
  }
  return lambda;
}

// The first element in A_i must be A_0: the lambda at which
// the 1-orbit stable fixed point achieves superstability
void cal_and_display_fei_constants(
    iter_map map, mpf_class (*x_bar)(mpf_class),
    std::vector<mpf_class> A_i) {
  // calculating alphas
  std::vector<mpf_class> alpha_cal;
  for (int i = 0; i < A_i.size() - 2; i++) {
    alpha_cal.push_back((A_i[i + 1] - A_i[i]) /
                        (A_i[i + 2] - A_i[i + 1]));
  }

  // deltas
  std::vector<mpf_class> d_i;
  for (int i = 1; i < A_i.size(); i++) {
    mpf_class local_max(x_bar(A_i[i]), PRECISION);
    mpf_class tmp;
    iterate(tmp, map, std::pow(2, i - 1), A_i[i],
            local_max);
    d_i.push_back(abs(local_max - tmp));
  }

  cout << "di" << endl;
  display(d_i);
  cout << "alpha" << endl;
  display(alpha_cal);

  std::vector<mpf_class> delta_cal;
  for (int i = 0; i < d_i.size() - 1; i++) {
    delta_cal.push_back(d_i[i] / d_i[i + 1]);
  }
  std::cout << "delta" << std::endl;
  display(delta_cal);
}

mpf_class one_third(mpf_class) { return mpq_class(1, 3); }

void find_feigenbaum_numerically(
    iter_map map, mpf_func max_func, mpf_class lambda_start,
    mpf_class end, mpf_class lambda_jump,
    mpf_class lambda_inc, vector<mpf_class> A_i,
    unsigned int start_cycle, mpf_class threshold,
    unsigned int stop_at) {
  while ((lambda_start < end) && (A_i.size() < stop_at)) {
    if (is_super_stable(map, max_func, lambda_start,
                        start_cycle, threshold, 1)) {

      lambda_start = poke_lambda_iterate(
          map, max_func, lambda_start, start_cycle,
          lambda_inc / 200, 100, 2);
      A_i.push_back(lambda_start);
      lambda_inc /= DELTA;
      lambda_jump /= DELTA;
      lambda_start += (lambda_jump * 0.9);
      threshold /= (ALPHA * 0.9);
      start_cycle += 1;
    }
    lambda_start += lambda_inc;
  }

  cout << "lambda" << endl;
  display(A_i);

  cal_and_display_fei_constants(map, max_func, A_i);
}

int main() {
  std::cout << std::fixed << std::setprecision(64);

  std::vector<mpf_class> A_i = {0.5};
  find_feigenbaum_numerically(
      logistic, [](mpf_class) -> mpf_class { return 0.5; },
      mpf_class(0.7, PRECISION), mpf_class(1, PRECISION),
      mpf_class(0.28, PRECISION),
      mpf_class(0.001, PRECISION), A_i, 1,
      mpf_class(0.005, PRECISION), 16);

  std::vector<mpf_class> A_i_skewed = {2.25};
  find_feigenbaum_numerically(
      logistic_skewed, one_third, mpf_class(2.8, PRECISION),
      mpf_class(6, PRECISION), mpf_class(2.8, PRECISION),
      mpf_class(0.001, PRECISION), A_i_skewed, 1,
      mpf_class(0.001, 128), 16);

  return 0;
}

\end{lstlisting}


