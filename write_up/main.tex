\documentclass[12pt,oneside]{report}

\usepackage[utf8]{inputenc}
\usepackage[backend=biber, sorting = none]{biblatex}

\addbibresource{./reference.bib}
% linktocpage shall be added to snippets.

% \usepackage{hyperref,theoremref}
% \hypersetup{
% 	colorlinks, 
% 	linkcolor={red!40!black}, 
% 	citecolor={blue!50!black},
% 	urlcolor={blue!80!black},
% 	linktocpage % Link table of content to the page instead of the title
% }

\usepackage{subfiles}
\usepackage{blindtext}
\usepackage{tikz}
\usetikzlibrary{cd}
\usetikzlibrary{positioning}

\usepackage{amssymb}
\usepackage[utf8]{inputenc}
\usepackage{mathtools}
\usepackage{titlesec}
\usepackage{amsthm}
\usepackage{thmtools}
\usepackage{amsmath}
\usepackage{graphicx}
\usepackage{titlesec}
\usepackage{xcolor}
\usepackage{multicol}
% \usepackage{hyperref}
\usepackage{import}
\usepackage{savetrees}

\usepackage{algorithm}
\usepackage{algpseudocode}
\usepackage[toc,page]{appendix}
\usepackage{listings}
\lstdefinestyle{cppstyle}{
    language=C++,
    basicstyle=\ttfamily\footnotesize,        % Small but readable font
    keywordstyle=\color{blue}\bfseries,       % Keywords in blue and bold
    stringstyle=\color{teal},                 % Strings in teal
    commentstyle=\color{gray}\itshape,        % Comments in italic gray
    morecomment=[l][\color{magenta}]{\#},     % Preprocessor directives in magenta
    numbers=left,                             % Line numbers on the left
    numberstyle=\tiny\color{gray},            % Small gray line numbers
    stepnumber=0,                             % Number every line
    frame=single,                             % Add a thin border around the code
    rulecolor=\color{black},                  % Border color
    tabsize=4,                                % Set tab width
    breaklines=true,                          % Enable line wrapping
    breakatwhitespace=true,                   % Break at spaces
    showspaces=false,                         % Hide space markers
    showstringspaces=false,                   % Hide spaces in strings
    captionpos=b,                             % Caption at bottom
    aboveskip=5pt,                            % Space before code block
    belowskip=5pt,                            % Space after code block
    backgroundcolor=\color{black!5},          % Light gray background
}

\lstdefinestyle{python}{
    language=Python,
    basicstyle=\ttfamily\footnotesize,        
    backgroundcolor=\color{black!5},    % Light gray background
    keywordstyle=\bfseries\color{blue}, % Bold blue keywords
    stringstyle=\color{teal},           % Teal strings
    commentstyle=\color{gray},          % Gray comments
    stepnumber=0,                             % Number every line
    frame=single,                             % Add a thin border around the code
    rulecolor=\color{black},                  % Border color
    tabsize=4,                                % Set tab width
    breaklines=true,                          % Enable line wrapping
    breakatwhitespace=true,                   % Break at spaces
    showspaces=false,                         % Hide space markers
    showstringspaces=false,                   % Hide spaces in strings
    captionpos=b,                             % Caption at bottom
    aboveskip=5pt,                            % Space before code block
    belowskip=5pt,                            % Space after code block
    backgroundcolor=\color{black!5},          % Light gray background
}


\usepackage{float}
\usepackage{subcaption}
\newcommand{\dd}[1]{\mathrm{d}#1}


\newtheorem{thm}{Theorem}[chapter]
\newtheorem{lemma}[thm]{Lemma}
\newtheorem{coro}[thm]{Corollary}
\newtheorem{prop}[thm]{Proposition}
\theoremstyle{definition}
\newtheorem{defn}[thm]{Definition}

\theoremstyle{definition}
\newtheorem{axiom}[thm]{Axiom}
\newtheorem{observation}[thm]{Observation}

\theoremstyle{remark}
\newtheorem{remark}[thm]{Observatio}
\newtheorem{hypothesis}[thm]{Coniectura}
\newtheorem{example}[thm]{Example}
\newtheorem{exmp}[thm]{Example}
\newtheorem{exampple}{Example}

\renewcommand\emptyset{\varnothing}
\renewcommand\mod{\text{ mod }}
%TODO mayby proof environment shall have more margin
% \renewenvironment{proof}{\vspace{0.4cm}\noindent\small{\emph{Demonstratio.}}}{\qed\vspace{0.4cm}}
% \renewenvironment{proof}{{\bfseries\emph{Demonstratio.}}}{\qed}
\renewcommand\qedsymbol{Q.E.D.}
% \renewcommand{\chaptername}{Caput}
% \renewcommand{\contentsname}{Index Capitum} % Index Capitum 
% \renewcommand{\emph}[1]{\textbf{\textit{#1}}}
\renewcommand{\ker}[1]{\operatorname{Ker}{#1}}

%\DeclareMathOperator{\ker}{Ker}

% New Commands
\newcommand{\bb}[1]{\mathbb{#1}} %TODO add this line to nvim snippets
\newcommand{\orb}[2]{\text{Orb}_{#1}({#2})}
\newcommand{\stab}[2]{\text{Stab}_{#1}({#2})}
\newcommand{\im}[1]{\text{im}{\ #1}}
\newcommand{\se}[2]{\text{send}_{#1}({#2})}
\newcommand{\be}{b_\text{eff}}
% \renewcommand{\L}{L_{\lambda}(x)}

% project specific macros
% m stands for maximum
\newcommand{\mx}{\overline{x}}

\date{7 March, 2025}
\title{Quantifying Chaos}
\author{Luc Dixon, Yuhao Han, Logan Hinkley}

\usepackage[y4project,fancyhdr,hyperref,colour]{edmaths}
\renewcommand{\L}{L_{\lambda}}
\flushbottom

\begin{document}
\maketitle

\pagenumbering{roman}

\chapter*{Acknowledgements}
\addcontentsline{toc}{chapter}{Acknowledgements}
We sincerely thank our supervisor, Prof. Jacques Vanneste, without whose support and guidance this report would have been impossible.

\begin{abstract}
Chaos is a universal phenomenon observed in every aspect of the physical world. Examples include weather systems, celestial movements, demographic evolutions, and economic activities. 
The descriptions thereof often comprise of the adjectives `unpredictable,' `disordered,' `entropic,' `confusing,' `subject to chances,' and, in summary, `chaotic.'
It may be of surprise that much more can be said to even the simplest examples.
In this report we will demonstrate there are rich, universal, and quantitative mathematical properties shared by all of these chaotic systems.
This report discovers the order within the chaos.

We begin by introducing examples of chaos and their defining characteristics. 
The intuitive descriptions of chaos, as will be shown, can be rigorously described by Lyapunov exponents.
Then, we move on to investigate a class of objects that are a natural result of chaos. These objects appear odd, and will require the quantifiable property of non-integer dimensions.
After this, the iterative logistic map as a model of population is studied in detail. The map exhibits an interesting behaviour that results in chaos that is also observed among a large class of similar systems. 
After developing some mathematically rigorous tools to study chaos, the final step is the derivation of Feigenbaum's universal constants, which are found to be shared across the wide class of systems we previously investigate.


\end{abstract}
\declaration
\dedication{Musa mihi causas memora ...\\
	Muse, tell me the causes ...
}

\tableofcontents
\newpage
\pagenumbering{arabic}

% \addcontentsline{toc}{chapter}{Introduction}
\chapter*{Introduction}
\addcontentsline{toc}{chapter}{Introduction}
The Oxford dictionary definition of chaos is `a complete lack of order'. 
In mathematics, the notion of chaos follows this definition closely and is the study of a dynamical system\footnote{
	The term \emph{dynamical system} describes a branch of mathematics that studies a process which evolves over time \cite{Devaney_green_book_chaos_definition}. 
	For all practical purpose a dynamical system can be regarded as a equation $F: T \rightarrow V$, where $T$ is the space of time and $V$ is a some system of interest. 
	Most of the dynamical systems in this report are one dimensional in discrete time interval generated by iterative maps. 
That is, there are $x_0, x_1, \cdots$ and some function $f$ such that $f(x_n) = x_{n+1}$, where $x_{i}$ denotes the system at time $i$.}.
We will postpone the rigorous definition of chaos to chapter \ref{chapter:chaos_with_rigor} while, for the moment, we demonstrate chaos with the example of the Lorenz system.


\begin{exmp}
	The Lorenz system\cite{lorenz1963deterministic}, consisting of three differential equations, \eqref{eq:lorenz} is a model for the atmospheric convection depending on three parameters: $\sigma, \rho$ and $\beta$.
	It is one of the earliest studied chaotic systems.
	\begin{equation}\label{eq:lorenz}
    \frac{dx}{dt} = \sigma (y-x), \quad \frac{dy}{dt} = x(\rho -z)-y \quad\text{and} \quad \frac{dz}{dt} = xy-\beta z.
	\end{equation}
	Although for given values of parameters and initial conditions, there exists a unique and deterministic solution, the system is described as chaotic, or `non-periodic' in Lorenz's words, in the sense that small perturbations in the initial conditions lead to vastly different trajectories.
	Lorenz describes this phenomenon with his famous quote of `butterfly effect': `Does the flap of a butterfly's wings in Brazil set off a tornado in Texas?'.
	This is demonstrated in Figure \ref{fig:lorenzrb}, which graphed the solution to the Lorenz system with $\sigma=28$, $\rho=10$ and $\beta=8/3$.

\begin{figure}
    \centering
    \includegraphics[width=\linewidth]{Images/lorenz red blue.png}
    \caption{The Lorenz `attractor' for parameter values $\sigma=28$, $\rho=10$ and $\beta=8/3$, traced out by the standard initial point $(x,\ y,\ z)=(1.0,\ 1.0,\ 1.0)$ in black. The trajectories of the initial points $(x,\ y,\ z)=(1.1,\ 1.0,\ 1.1)$ and $(x,\ y,\ z)=(1.0,\ 1.1,\ 1.0)$ after $1,000$ iterations are shown in red and blue, respectively. The region on the centre-left side of the attractor (that appears purple) consists of the earlier points on the trajectories where they are still nearby each other; the fact that, elsewhere, we see distinct red and blue spirals, is a great example of chaotic behaviour.}
    \label{fig:lorenzrb}
\end{figure}
\end{exmp}

Lorenz's description of chaos is qualitative and intuitive. 
The goal of this report, however, is to demonstrate this intuition precisely and rigorously via theorems, proofs, numerical computations, and visualisations.
Specifically, we will look at: the Lyapunov exponent, which will show us how sensitive a system is to its initial conditions; attractors and their fractal dimension, telling us how a system approaches a set of points in phase space; the logistic map, a simple system that showcases chaos, which will be our go-to example in many cases; and Feigenbaum's universal constants, numbers that reappear throughout a wide class of chaotic systems. 
Each of these topics will call for particular definitions of chaos, which vary in rigour, so we will investigate the simpler topics first and introduce the more rigorous definitions as needed.

While most of the theorems and definitions come from the standard literature on dynamical systems, all proofs, numerics, and figures are original.

\chapter{Lyapunov Exponents}
\section{Definitions and Numerical Computations of Lyapunov Exponents}

One of the key defining properties of chaos in dynamical systems is that small perturbations in the initial conditions leads large difference overtime.
With the assumption that the distance of nearby points increases exponentially over time, the following simplistic formula for the dynamical system of an iterative map $F$ can be deduced
\begin{align}\label{eq:lyapunov_exponent}
    \left|F^n(x_0+\epsilon)-F^n(x_0)\right| \sim \epsilon e^{n \lambda}.
\end{align}
where $\lambda$ is the Lyapunov exponent \cite{nonlinear_system} \cite{lyapunov}, $x_0$ the point at discrete time $0$ and evolves by the relations $x_{n+1} = F(x_n)$.

A positive $\lambda$ means nearby points of $x_0$ will move away from it as time progresses, and a negative $\lambda$ means nearby points will converge to $x_0$. 
The absolute value of $\lambda$ denotes the rate of diverging or converging.

The existence of such $\lambda$ is not obvious, the analytical evaluation of $\lambda$ is often hopeless, and the assumption seems very questionable.
However, if acquiescing such a formula, $\lambda$ can be calculated numerically and give important insights of the dynamical system. 


Assuming that $\epsilon e^{nL} \to 0$ and $\epsilon$ is small, \eqref{eq:lyapunov_exponent} becomes
$$
\epsilon \left| \frac{dF^n(x_0)}{dx} \right| \sim \epsilon e^{nL}.
$$

By taking logarithm
\begin{align}
    \lambda 
    &= \lim_{n \to \infty}\left(\frac{1}{n}\ln{\left|\frac{dF^n(x_0)}{dx}\right|}\right)  \\
    &= \lim_{N \to \infty}\left\{\frac{1}{N}\sum_{n=0}^{N-1}\ln{|F'(x_n)|}\right\}  \label{eq:lambda}.
\end{align}
Where in the last step we have labelled the points $x_1 = F(x_0), x_2 = F(x_1), \dots$ and applied the chain rule.
All of the numerical calculations of Lyapunov exponents in this report will use formula \eqref{eq:lambda}.

The equation \eqref{eq:lambda} fails to make sense at the points $x$ where $F'(x) = 0$.
Such points are \emph{superstability} points, and the Lyapunov exponents of which are defined as $- \infty$.

The following are some examples of calculating Lyapunov exponents.

\begin{exmp}
	Let us the tent map defined in equation \eqref{eq:tent} and the dynamical system depending on one parameter $r$ generated by the iterations of it.
    \begin{align}
        F(r, x)= 
        \begin{cases}
            r x, & \text{ if } x \in [0,1/2) \\
            r (1-x), & \text{ if } x \in [1/2,1]
        \end{cases} \label{eq:tent}
    \end{align}
	By equation \eqref{eq:lambda}, the Lyapunov exponent at $0$ is $\ln r$, which is negative if $r < 1$ and positive otherwise.
	The higher subfigure of \ref{fig:lyapunov_tent} is the bifurcation diagram of the Tent map showing its dynamical properties. 
	It is obtained by selecting 1000 equally spaced $r \in [0,2]$, and for each $r$ picking a random starting point $x_0$, iterating $x_0$ for $1000$ times, ignoring the first $200$ iterations, and plotting the rest $800$ $ x_i$ as a faint blue dot at coordinate $(r, x_i)$. 
	There is a dramatic change in behavior exactly at $r=1$ where the Lyapunov exponent changed from negative to positive, before which all the $x_0$ converges to $0$, whereas afterwards the system shows an apparent chaotic behavior by casting an interesting homogeneous shape of doubly curved triangle.

    \begin{figure}
        \centering
        \includegraphics[width=\linewidth]{Bifurcation Images/Bifurcation_tent.png}
        \includegraphics[width=\linewidth]{Bifurcation Images/Lyapunov_Tent.png}
        \caption{Bifurcation diagram and Lyapunov exponent plot for the tent map \eqref{eq:tent}}
        \label{fig:lyapunov_tent}
    \end{figure}
\end{exmp}

\begin{exmp}
	Similarly consider the dynamical system depending on one parameter $r$ generated by the iterations of the map
    $F(r,x)=x^2+r$ where $x,\lambda \in \mathbb{R}$.
	Unlike the case of tent map, there is no simple closed form for Lyapunov exponent, and computer simulations are required. 
	The bifurcation diagram and the Lyapunov exponent calculated with formula \eqref{eq:lambda} is shown in figure \ref{fig:lyapunov_x^2}.
    \begin{figure}
        \centering
        \includegraphics[width=1\linewidth]{Bifurcation Images/bifurcation_x^2.png}
        \includegraphics[width=1\linewidth]{Bifurcation Images/lyapunov_x^2.png}
        \caption{Bifurcation diagram and Lyapunov exponent plot for the map $F(\lambda,x)=x^2+\lambda$}
        \label{fig:lyapunov_x^2}
    \end{figure}
	A very interesting phenomenon of periodic doubling is apparent from the bifurcation diagram (which is studied in detail in chapter \ref{chapter:bifurcation}).
	Similar to the tent map, regions when the Lyapunov exponent is negative corresponds to distict lines in the bifurcatio diagram, while those with positive Lyapunov exponent corresponds to chaotic bands.
\end{exmp}




\section{Lyapunov Exponents for a Continuous Time Series}

Previously we have discussed the implications and how we solved for the Lyapunov exponent in discrete dynamical systems. 
But, if we consider continuous time dynamical systems in multiple dimensions, we observe something different. 
For continuous systems we need to consider the complete spectrum of lyapunov exponents which utilizes spectral calculations. 
Given a continuous dynamical system in an $n$-dimensional phase space, we monitor the long-term evolution of an infinitesimal $n$-sphere of initial conditions. 
It quantifies the average exponential growth or decay rates of perturbations in different directions in phase space. 
For an $n$-dimensional system, the Lyapunov spectrum consists of $n$ exponents, $L_1 \geq L_2 \geq \dots \geq L_n$, which describe the system's behavior in terms of expanding, neutral, and contracting directions.

Lets consider a continuous-time dynamical system defined by
$$
\dot{\mathbf{x}}(t) = \mathbf{f}(\mathbf{x}(t)),
$$
where $\mathbf{x}(t) \in \mathbb{R}^n$ is the state vector, and $\mathbf{f}$ is a smooth vector field. To study the evolution of perturbations, we linearize the system around a reference trajectory $\mathbf{x}(t)$. The evolution of a small perturbation $\delta\mathbf{x}(t)$ is governed by the linearized dynamics:
$$
\dot{\delta\mathbf{x}}(t) = \mathbf{J}(\mathbf{x}(t)) \delta\mathbf{x}(t),
$$
where $\mathbf{J}(\mathbf{x}(t))$ is the Jacobian matrix of $\mathbf{f}$ evaluated at $\mathbf{x}(t)$:
$$
\mathbf{J}(\mathbf{x}(t)) = \frac{\partial \mathbf{f}}{\partial \mathbf{x}} \bigg|_{\mathbf{x}(t)}.
$$
The $i$-th Lyapunov exponent $L_i$ is thus defined as:
\begin{align}
    L_i = \lim_{t \to \infty} \frac{1}{t} \log_2 \frac{p_i(t)}{p_i(0)}, \label{eq:continuous}
\end{align}
where $p_i(t)$ is the length of the $i$-th principal axis of an infinitesimal $n$-sphere of initial conditions evolved under the flow. The Lyapunov exponents describe the long-term exponential growth rates of perturbations in different directions.

An infinitesimal $n$-sphere of initial conditions evolves into an $n$-ellipsoid due to the locally deforming nature of the flow. The principal axes of the ellipsoid grow or shrink at rates determined by the Lyapunov exponents. A positive exponent ($L_i > 0$) indicates expansion in the corresponding direction. A zero exponent ($L_i = 0$) indicates a neutral direction, such as along the flow of a trajectory. A negative exponent ($L_i < 0$) indicates contraction in the corresponding direction.

From \eqref{eq:continuous} we can see that the ellipsoid grows with a rate of $2^{L_1t}$. The area is defined by the first 2 principle axis $2^{(L_1+L_2)t}$, and the volume by the third, $2^{(L_1+L_2+L_3)t}$. Thus in $n$-dimensional space volume of the ellipsoid evolves as
$$
V(t) \sim 2^{(L_1 + L_2 + \dots + L_n)t}.
$$
For dissipative systems, a system that releases energy instead of retaining it, the sum of the Lyapunov exponents is negative, indicating that the volume of the ellipsoid contracts over time.

In 3 dimensions, the sum of all the Lyapunov exponents is equivalent to the trace of the systems corresponding Jacobian matrix. 
The possible Lyapunov spectra and the corresponding attractors are $(+,0,-)$ which signifies a strange attractor or chaotic behavior. 
In order for the system to be chaotic only one of the individual exponents needs to be positive. 
$(0,0,-)$ tells us that the system will demonstrate quasi-periodic behavior. 
$(0,-,-)$ corresponds to the presence of a limit cycle or periodic behavior. 
Finally $(-,-,-)$ telling us that there is a stable fixed point. 
The sum of the Lyapunov exponents equals the time-averaged divergence of the phase space velocity. 
The value indicates the divergence of the flow and is the fractional rate of the volume expansion or contraction of the system, but for a conservative or Hamiltonian system this sum is zero. 
For dissipative systems, this sum is negative, ensuring that at least one exponent is negative. 
As a result, the long-term motion of trajectories converges to a limit set with zero volume, known as an attractor. 
While Lyapunov exponents are defined based on long-time averages, short trajectories may not fully capture the distinct behaviors associated with positive, zero, and negative exponents.

% To compute the Lyapunov spectrum numerically, we start by linearising the system by solving for the Jacobian matrix $\mathbf{J}(\mathbf{x}(t))$ along the reference trajectory $\mathbf{x}(t)$. We then need to evolve the perturbation vectors. We need to initialize a set of $n$ orthonormal perturbation vectors $\{\mathbf{v}_1, \mathbf{v}_2, \dots, \mathbf{v}_n\}$ and evolve them using the linearised dynamics
% $$
% \dot{\mathbf{v}}_i(t) = \mathbf{J}(\mathbf{x}(t)) \mathbf{v}_i(t).
% $$
% Subsequently we need to use the Gram Schmidt reorthonormnalisation on the vector frame that we created in order to to maintain their independence and prevent numerical overflow. Then we can track the growth rates of the perturbation vectors and accumulate their logarithmic growth rates
% $$
% \sum_{i=1}^n \ln \|\mathbf{v}_i(t)\|.
% $$
% where the Lyapunov exponents are computed as the time-averaged logarithmic growth rates
% $$
% L_i = \lim_{t \to \infty} \frac{1}{t} \ln\frac{\|\mathbf{v}_i(t)\|}{\|\mathbf{v}_i(0)\|}.
% $$


The Lyapunov spectrum is closely related to the fractal dimension of the attractor. The Kaplan–Yorke conjecture uses Lyapunov exponents to deduce the dimension of an attractor within a dynamical system. In order to quantify the complexity of chaotic attractors, which often exhibit fractal structure, Lyapunov exponents are able help provide information about the system's dynamics, while fractal dimension characterises the geometry of the attractor. The conjecture is able to link these two concepts together. The conjecture is based on the idea that the sum of Lyapunov exponents relates to the expansion and contraction rates of phase space volumes. The fractal dimension is then estimated by balancing the cumulative expansion (positive Lyapunov exponents) against the cumulative contraction (negative Lyapunov exponents). Let us arrange the Lyapunov exponents from largest to smallest such that $L_1 \geq L_2 \geq \dots \geq L_n$. Let $k$ thus be the largest index in which
$$
\sum_{i=1}^k L_i\geq 0 \quad\quad \text{ and } \quad \quad \sum_{i=1}^{k+1} L_i <0
$$
The Kaplan-Yorke dimension, defined as
$$
D = k + \frac{\sum_{i=1}^k L_i}{|L_{k+1}|},
$$
where $k$ is the largest integer such that $\sum_{i=1}^k L_i \geq 0$ which provides an estimate of the fractal dimension.

\begin{exmp}
    \begin{figure}
        \centering
        \includegraphics[width=0.95\linewidth]{Bifurcation Images/lyapunov_henon.png}
        \caption{Caption}
        \label{fig:lyapunov_henon}
    \end{figure}
    Lyapunov exponents base e: 0.41762663136121125 -1.6215994356871315
    Lyapunov exponents base 2: 0.6025078700079933 -2.339473464174202
\end{exmp}

% Given a continuous dynamical system in an $n$-dimensional phase space, we monitor the long-term evolution of an infinitesimal $n$-sphere of initial conditions. The sphere will become an $n$-ellipsoid due to the locally deforming nature of the flow. The $i$th one-dimensional Lyapunov exponent is then defined in terms of the length of the ellipsoidal principal axis $p_i(t)$
% \begin{align}
%     L_i=\lim_{n \to \infty} \frac{1}{n}\log_2\frac{p_i(t)}{p_i(0)} \label{eq:continous}
% \end{align}
% where $L_i$ are the individual exponents. Thus the Lyapunov exponents are related to the expanding or contracting nature of different directions in phase space. Since the orientation of the ellipsoid changes continuously as it evolves, the directions associated with a given exponent vary in a complicated way through the attractor. One cannot, therefore, speak of a well-defined direction associated with a given exponent. From \eqref{eq:continous} we can see that the ellipsoid grows with a rate of $2^{L_1t}$. The area is defined by the first 2 principle axis $2^{(L_1+L_2)t}$, and the volume by the third$2^{(L_1+L_2+L_3)t}$. From this we can create a new definition for the spectrum of exponents, the sum of the first $j$ exponents is define by the long term exponential growth rate of a $j$-volume element. This will provide us with a basis for the spectra technique fro experimental data.

% Any continuous time dependent dynamicla system without a fixed point will have at least one zero exponent which links with the slowly changing magnitude of a principle axis  tangent to the flow

% In a three-dimensional continuous dissipative dynamical system the only possible spectra, and the attractors they describe, are as follows: $( + , 0 , - )$ , a strange attractor; $(0,0,-)$, a two-toms; $(0, - , -)$, a limit cycle; and $( - , - , - )$ , The sum of each individual exponents is the time-averaged divergence of the phase space velocity; hence any dissipative dynamical system will have at least one negative exponent, the sum of all of the exponents is negative, and the post transient motion of trajectories will occur on a zero volume limit set, an attractor. Since Lyapunov exponents involve long-time averaged behavior, the short segments of the trajectories shown in the figure cannot be expected to accurately characterize the positive, zero, and negative exponents; nevertheless, the three distinct types of behavior are clear.

% The Lyapunov spectrum is closely related to the fractional dimension of the associated strange attractor.
%--------------------------------------------------------------------------------------------------%

% \textbf{Show that the $p$-cycle $\{X_1, X_2, \dots, X_p\}$ of the continuously differentiable map $F: \mathbb{R} \to \mathbb{R}$ is stable if  
% \[
% | F'(X_1) F'(X_2) \dots F'(X_p)| < 1.
% \]
% This is a proof the textbook refers to a lot in the section and I dont know how to interpret it or include it.}


% Idea: If we have an initial condition $x_0$ which will generate our sequence for a distcete dynamical system to use discrete time, and a point nearby $x_0 + \epsilon_0$. Let $\epsilon_n$= separation of orbits from $x_0$ and orbit from $x_0 + \epsilon_0$

% If $|\epsilon_n| \sim |\epsilon_0|e^{n\lambda}$ is converging exponentially, $\lambda$ is they lyanopunov exponent
% -$\lambda>0$ means chaos
\chapter{Fractal Dimension}
%Chaotic systems, when visualised, often exhibit complex patterns due to their non-linear behaviour. These objects can loosely be defined as fractals\footnote{There are many definitions for what a fractal is. Mandelbrot defines them as (...)}, and we can measure how complicated one fractal is compared to another. Though our notion of chaos is not necessarily relevant in this section, fractal dimension will be used to measure the complexity of chaotic systems in further sections.%

%Fractal dimension can be summarised as the measure of an object's complexity or detail. Unlike the dimension of standard objects, fractal dimensions can take non-integer values. This allows objects in the same space to be compared in terms of how complicated their shape is. There are multiple ways to calculate or approximate an object's fractal dimension%
When studying dynamical systems that exhibit chaotic behaviour, we often choose to reduce them to a simpler form in order to obtain quantifiable properties. One key property we look for is \textbf{fractal dimension}, which can be thought of as a set's complexity or detail. It can also be seen as a measure of how an object fills space at small scales. Formally, fractal dimension is an index of sets that characterises their complexity as a ratio of the change in detail to the change in scale \cite{mandelbrot1983fractal}. Unlike the values of dimension we are used to, fractal dimension can take non-integer values to distinguish between two objects that fill the same space but in different ways. For instance, a curve in $\mathbb{R}^2$ with fractal dimension close to one behaves much like a standard curve, while one with a fractal dimension close to two behaves more like a surface, e.g. a space-filling curve. We can reduce a dynamical system to the set of points that its trajectories tend toward and find the fractal dimension of this set, giving us an idea of how the system behaves.

\section{Fractals}
There are many definitions for what a fractal really is, many of which are not rigorous and can only be applied in specific cases. Our definition will be that of Benoît Mandelbrot's \cite{mandelbrot1983fractal}.
\begin{defn}
    Let $X$ be a set. We say that $X$ is a fractal if its fractal dimension strictly exceeds its topological dimension.
\end{defn}
In this report, the `topological dimension' will be equal to the standard Euclidean dimension we expect of objects. We have been vague in introducing the notion of non-integer values of dimension; this is because different methods of calculating fractal dimension are used depending on the context of the object. We will introduce these methods after the different contexts are clear.

\section{Self-Similarity and the Mass-Scaling Factor Method}
Our first context is one which is often used as a definition for fractals; that is, self-similar objects. Intuitively, \emph{self-similar} objects are those that are made up of smaller copies of themselves, such as a solid cube. We see a sense of self-similarity in nature, so fractals are often exemplified as, say, the branches of a tree. While fascinating, this notion of self-similarity will not be relevant in our case. We will see sets that are infinitely perfectly self-similar as they have been defined to be this way by an unbounded recursive construction.

While a little more abstract, the fractal dimensions of objects that exhibit self-similarity can be calculated in a simple way. The following method is motivated by translating our usual notions of an object's dimension to that of fractal dimension. 

\begin{prop} \label{MSF}
    The mass-scaling factor method:
    \begin{enumerate}
        \item Say some self-similar object has a mass equal to one.
        \item Within the object, find a section that is an identical copy of the original.
        \item Comparing the copy to the original, work out the mass $m$ and scaling factor $s$ of the copy (this should be clear for simple self-similar shapes).
        \item Let $M=1/m$ and $S=1/s$. The fractal dimension $D$ of the object is then $S^D=M.$ Equivalently, $D=\log{_SM}$
    \end{enumerate}
\end{prop}
%\begin{defn}
    %The \textbf{scaling factor}, $S$ of a self-similar object is the ratio in size between itself and the `first' copy of itself.
%\end{defn}

%\begin{defn}
    %Suppose a self-similar object has `mass' equal to one. The \textbf{mass scaling factor}, $M$, of the object is the ratio between the masses of the object and the first copy of itself.
%\end{defn}
%\textbf{What do we mean by mass?}
%\begin{defn}
    %The \textbf{fractal dimension} of an object is the real number, $D$, such that $$S^D=M.$$ Equivalently, $$D=\log{_SM}$$
%\end{defn}
This definition of an object's dimension satisfies our usual methods of determining dimension; using elementary intuition, we can see that a straight line, a square, and a cube have fractal dimensions one, two, and three, respectively. In fact, the idea of dimension being a measure of how an object fills space has always been true for normal notions of dimension; scaling down a cube by some factor reduces its mass by that factor to the power of three - the dimension of the cube. As expected, this means they are not fractals due to the fractal dimension being equal to the topological dimension. However, what we have now is a distinction between, say, two curves in $\mathbb{R}^2$ or two surfaces in $\mathbb{R}^3$, where the higher fractal dimension points towards a more detailed structure.
%\begin{exmp}
    %The `Koch curve' is constructed by repeatedly adding triangular bumps to the centre-thirds of a line, starting with a straight line of unit length.
    
    %If we take a straight line and the Koch curve embedded in $\mathbb{R}^2$, both are continuous curves, so they are of the same dimension in the usual sense; however, we will see that the fractal dimensions are not equal due to the Koch curve being more complex. 
    %\begin{figure}
        %\centering
        %\includegraphics[width=0.8\linewidth]{Images/Koch.png}
        %\caption{Straight line and Koch Curve}
       % \label{fig:enter-label}
    %\end{figure}
    %Both are self-similar, since any cut of a line is similar to the line and the Koch curve splits into four identical copies of itself, each of which is scaled down in size by $1/3$. It is clear that, taking any cut of the straight line, say $1/a^{th}$ of it, gives a line of mass $1/a$, so we have that the line has a scaling factor equal to its mass-scaling factor, giving a fractal dimension of one, as we would expect. For the Koch curve, we have a scaling factor of $1/3$, since each copy of itself is $1/3^{rd}$ the length of the original and a mass-scaling factor of $1/4$, since we have four copies of this size. Therefore, we have that the fractal dimension of the Koch curve is given by the value of $D$ such that
    %$$\frac{1}{3}^D=\frac{1}{4},$$
    %equivalently, such that
    %$$3^D=4.$$
    %Therefore,
    %$$D=\log{_34}\approx1.26186.$$
    %This means that the Koch curve is a fractal, since its fractal dimension is strictly greater than its topological dimension which, being a curve in the plane, is one.
%\end{exmp}

\begin{exmp}
    The `Koch curve', pictured in Figure \ref{fig:Koch}, is constructed in the following way. Take a straight line segment embedded in $\mathbb{R}^2$ and remove the middle third and replace it with an `equilateral-triangle bump'. Note there are now four straight lines of length $1/3$ of the original. On each new straight line, add equilateral-triangle bumps to their middle thirds and repeat. At each stage, each quarter of the curve is an identical copy of the last version, scaled down by $1/3$.
    \begin{figure}
        \centering
        \includegraphics[width=1\linewidth]{Images/Koch new.png}
        \caption{Construction of the Koch curve, showing the first, second and fifth iterate. In blue are the sections that are scaled copies of the iterate before it.}
        \label{fig:Koch}
    \end{figure}
    Let us follow the steps in Proposition \ref{MSF} to determine the fractal dimension of the Koch curve:
    \begin{enumerate}
        \item We say the Koch curve has mass and length one.
        \item Considering the curve after infinite iterates, each quarter of the curve is identical to the entire thing.
        \item Since each quarter of the curve is a copy of the original scaled down by $1/3$, one copy has mass $m=1/4$ and scaling factor $s=1/3$.
        \item We have $M=1/(1/4)=4$ and $S=1/(1/3)=3$, so the fractal dimension of the Koch curve is $D=\log_34\approx1.26186$
    \end{enumerate}
    Since the Koch curve is a continuous curve in $\mathbb{R}^2$, its topological dimension is one. Therefore, we have that the Koch curve is indeed a fractal, as its fractal dimension strictly exceeds its topological dimension.
\end{exmp}
%From the example above, we can see that we have a measurable value that will distinguish shapes based on their complexity. This will prove useful in chaos theory as knowing the fractal dimension of an object then allows us to talk about measures, which will be affected by the complexity of objects. The issue will be for objects that are not so simply defined and do not exhibit self-similarity. For this, we will need another method for calculating fractal dimension%.
We see here, a glimpse of what an object's fractal dimension is, with the restriction that our method only works in the case of self-similarity. We have another method that does not depend on this property; however, the context in which it is used must first be introduced.
\section{Attractors and The Box-Counting Method}
We often want to look for the points that dynamical systems tend toward as time progresses. In our models, the existence and uniqueness theorem tells us that these points are deterministic in that fixed initial conditions will grant the same result every time. For stable systems, we can further say that some small perturbation of the initial conditions will not greatly affect the solution; this allows us to make strong predictions about how the system will evolve over time. With this in mind, we naturally arrive at the definition of an \emph{attractor}, the set of points in which a system evolves toward over time and then stays there forever. A key property of attractors is that they do not change under a small perturbation of initial conditions, the trajectories will eventually fall into the same set of points.

%\begin{defn}
    %An \textbf{attractor}, $\mathbf{X}  \in \mathbb{R}^n$, of an $n$-dimensional dynamical system, $F(\mathbf{x},t)$, is the set of points $\mathbf{x}_\infty\in\mathbb{R}^n$ that the trajectory of an initial point $\mathbf{x_0}$ tends toward $t\to\infty$.
    %$$\mathbf{X} := \{\mathbf{x}_\infty\in\mathbb{R}^n:F(\mathbf{x_0},t)\to\mathbf{x}_\infty \text{ as } t\to \infty\},$$
    %such that points in some small neighbourhood of %$\mathbf{x_0}$ will share the same attractor%.%
%\end{defn}%

%\begin{defn}
    %Let $F(\mathbf{x},t)$ be an $n$-dimensional dynamical system and $\mathbf{x_0}$ be some initial point. An \textbf{attractor}, $\mathbf{X}  \in \mathbb{R}^n$ of $\mathbf{x_0}$ is the set of points
    %$$\mathbf{X} := \{\mathbf{x}\in\mathbb{R}^n:F(\mathbf{x_0},t) = \mathbf{x}\},$$
    %such that the trajectories are dense near these points after some large $t$. Furthermore, points in some small neighbourhood of $\mathbf{x_0}$ share the same attractor.
%\end{defn}

%\begin{defn}
    %Let $F(\mathbf{x},t)$ be an $n$-dimensional dynamical system and $\mathbf{x_0}$ be some initial point. An \textbf{attractor}, $\mathbf{X}  \in \mathbb{R}^n$ of $\mathbf{x_0}$ is the set of points such that:
    %\begin{enumerate}
        %\item the trajectory remains dense in after some large number of iterations; and
        %\item points in some small neighborhood of $\mathbf{x_0}$ share the same attractor.
    %\end{enumerate}. 
%\end{defn}
%For simpler stable systems, we can find formulas for defining these sets by taking derivatives of the map $F$ and finding the stable points.

For stable systems, two types of attractors are: fixed points, a state of equilibrium; and limit cycles, a set of points that the system will stay in, periodically, forever. 
%Since we have a set of distinct points in $\mathbb{R}^n$, we say these attractors are one-dimensional. 
%Small changes to the initial conditions will have a small effect on the attractors for stable systems. If we alter the parameters of a given system, we can see changes in stability as well as changes in the attractor type. 

%Given a system that is not stable, the trajectories may diverge. In a confined space, we have the example of a fluid mixing in a box, and we say that the attractor of this system is the whole three-dimensional space.
For chaotic systems, things are different. Since we have a strong sensitivity to initial conditions, even stable points can become unstable at the slightest perturbation due to the existence of a positive Lyapunov exponent. This means we cannot define the set so easily since a minor change to one initial point could greatly affect its outcome over time. This being said, we may not have complete divergence since we can plot the trajectories after a high number of iterations and still see a distinct set. Furthermore, zooming in to these attractors, we see that they can take the form of fractals. 

These properties have coined the term `strange' attractors\cite{feldman2012chaos} since they paradoxically show convergence of a system that seems to never settle down. Thus the term \emph{strange attractor} describes attractors of dynamical systems that observe a sensitive dependence to initial conditions and a fractal structure. To better understand strange attractors, a staple example is the Hénon map\cite{nonlinear_system}, a simplified section of the famous Lorenz system.
\begin{exmp}
    We define the Hénon map, which acts on points $(x,y)\in \mathbb{R}^2$, as the discrete-time dynamical system, 
    \[
x_{n+1} = 1 - a x_n^2 + y_n, \quad y_{n+1} = b x_n.
    \]
    The classic parameters are $a=1.4$ and $b=0.3$, with which the map is chaotic. Plotting the trajectory of the initial point $(0,0)$ gives us a prime example of a strange attractor.
    \begin{figure}
        \centering
        \includegraphics[width=0.4\linewidth]{Images/Henon attractor.png}
        \caption{$100,000$ iterations of the Hénon map with initial point $(x_0, y_0)=(0,0)$}
        \label{fig:Henon1}
    \end{figure}
    We see the set of points this system attains as the nested curves in Figure \ref{fig:Henon1}. It is important to understand these curves are made up of distinct points, given by each iteration of the Hénon map, and these points are by no means following the curves continuously; we see that the 95th through 99th iterations\footnote{Choosing the 95th through 99th iterations was completely arbitrary in the interest of a diagram that represented the point well. Almost all iterations we could have chosen would have shown the sensitive dependence on initial conditions.} are far apart from each other and follow no obvious pattern. Changing the initial conditions by a small amount, we see that the attractor remains similar, but these same iterations appear in drastically different positions on it, alluding to the chaotic nature of this system. This behaviour can be seen in Figure \ref{fig:Henon2}.
    \begin{figure}
        \centering
        \includegraphics[width=0.75\linewidth]{Images/Henon attractor with labels.png}
        \caption{The Hénon attractor, generated from two different initial points, $(x_0,y_0) = (0,0)$ and $(x_0',y_0') = (0.5,0.5)$. The attractor is identical for both initial points while the system itself appears to be chaotic due to the iterates following drastically different trajectories.}
        \label{fig:Henon2}
    \end{figure}
    The fact we see a similar attracting set for both initial points indicates this is indeed an attractor by definition. We can give further evidence that this is a strange attractor by zooming in to a section of it and noticing that more and more curves appear the further we zoom. We see more curves appear in Figure \ref{fig:Henon3}. This intricate detail, no matter the scale, is a direct implication that this attractor is indeed a fractal.
    \begin{figure}
        \centering
        \includegraphics[width=1\linewidth]{Images/henon zoom.png}
        \caption{Zooming in to the blue boxes on the Hénon attractor to see the fractal structure of the object.}
        \label{fig:Henon3}
    \end{figure}
    Since the attractor has a sensitive dependence on initial conditions and a fractal structure, we conclude that this indeed is a strange attractor.
\end{exmp}
From this example, it is clear that the mass-scaling factor method of determining fractal dimension will no longer be of use, since we cannot simply find a smaller version of the Hénon map inside itself, let alone determine its mass-scaling factor. However, the idea of how the attractor fills space will motivate a new method for approximating the value.\\

Motivated by the need for a universal method for calculating fractal dimension, the following method can be used to approximate the fractal dimension of any object in some subset of $\mathbb{R}^n$. The idea is, given some complicated object, we measure the amount of space it fills as we zoom into its structure. We do this by overlaying a grid of $n$-dimensional boxes over the object and counting how many boxes it is contained in. We then decrease the size of the boxes and count again, repeating this process a number of times. With the results, we plot the logarithms of the number of boxes against the size of the boxes and take the slope of best fit to be the fractal dimension. Explicitly, we have a formula
\begin{equation}
    D = \lim_{\epsilon \to 0} \frac{\log N(\epsilon)}{\log(1/\epsilon)},
\end{equation}\label{boxcountingD}
where $\epsilon$ is the size of each box (length of one side of one $n$-cube) and $N(\epsilon)$ is the number of boxes that the object is contained in for each $\epsilon$. For complicated objects, the best way to do this is by use of computer code that will do this whole process to a strong level of accuracy. It must be noted that formula \ref{boxcountingD} may not be valid in practice, since it is only realistic to plot a finite number of points in Python. Therefore, we must be careful in ch, or else the box-count will `bottom-out' . We return to the previous example to calculate the fractal dimension of the Hénon attractor by the box-counting method.
\begin{exmp}
    Using Python, we calculate the fractal dimension of the Hénon attractor by following the box-counting method. The code overlays a grid of boxes over the fractal and counts the number of boxes that contain it as the box size decreases. In Figure \ref{fig:Henon4}, we see some blue boxes that contain the attractor as their size decreases.
    \begin{figure}
        \centering
        \includegraphics[width=1\linewidth]{Images/Henon boxes.png}
        \caption{Blue squares are those from the grid that contain at least one iterate of the Hénon attractor after $10,000,000$ iterations, using initial point $(0,0)$. The first $10,000$ iterates are discarded to remove any `dust' for accuracy.}
        \label{fig:Henon4}
    \end{figure}
    The code then plots $\log (N(\epsilon))$ against $\log (1/\epsilon)$ for many values of $\epsilon$ and finds the line of best fit through the points on the plot. The gradient of the line is then calculated to achieve a value for $D$. This is all given in Figure \ref{fig:Henon5}.
    \begin{figure}
        \centering
        \includegraphics[width=0.5\linewidth]{Images/henon loglog.png}
        \caption{$\log (N(\epsilon))$ as a function of $\log(1/\epsilon)$ for ten values of $\epsilon$ and the line of best fit through the data points which has gradient equal to the fractal dimension of the Hénon attractor.}
        \label{fig:Henon5}
    \end{figure}
    From the gradient of the line, we get a value of $D\approx1.264$, which is our fractal dimension of the Hénon attractor by the box-counting method. The code written for this example is shown in \ref{boxalg}.
\end{exmp}
With this, we have a reliable method for calculating the fractal dimension of a fractal object in $\mathbb{R}^n$. This gives an index of fractals, distinguished by their small-scale complexity. 
This complexity and these fractal shapes are seen throughout nature as a characteristic of what we consider to be a natural-looking shape\footnote{“Clouds are not spheres, mountains are not cones, coastlines are not circles, and bark is not smooth, nor does lightning travel in a straight line.”
― Benoît Mandelbrot \cite{mandelbrot1983fractal}}.
Noticing this, computer image generation uses fractals to create shapes that look natural, such as mountain ranges, plants, and coastlines. Without fractals, we could do our best to draw the detailed ridge of a mountain range from a given distance away, but the detail would not remain as we move closer. Fractals allow us to generate this detail to any scale; which, in the interest of lifelike simulation, makes for a much better representation of such objects. The fractals we use depend on what we are trying to model; mountain ranges, plants, and coastlines all have different ranges of fractal dimensions that best represent their real-life structure. 

%\section{Fractal Dimension Implications}
%For chaotic dynamical systems, we can now quantify the dimension of the attractors with values that better represent the strange structure they can have. These values give us an idea of the properties of the system itself.\\

%First, the higher the fractal dimension, the more an attractor of a system fills the space it occupies. In an $n$-dimensional subset of $\mathbb{R}^n$, this means that an attractor with a fractal dimension close to $n$ will be associated with a system that has a higher number of effective degrees of freedom.
%In particular, suppose we are working in some subset of $\mathbb{R}^2$. We have, on one extreme, that the attractor of the system is a single fixed point, where we would have a zero-dimensional attractor. On the other extreme, the attractor is the whole space, two-dimensional. A one-dimensional attractor would be one that takes the form of a smooth continuous curve\footnote{For discrete time systems, an attractor in the shape of a smooth curve will not be continuous in $\mathbb{R}^2$ but the box-counting dimension will still be equal to one.}. An attractor's fractal dimension then tells us how chaotic a system is; whether it behaves more like a fixed point or limit cycle, alluding to a system that is close to being stable; fills nearly the entire subspace, much like fluid mixing in a box; or it takes values across a set similar to a smooth curve, like the Hénon attractor.  How close the fractal dimension is to the integer values generally points towards how strong the attractor is, i.e., how long it takes for nearby points to get within a small distance of the attractor. This correspondence gives a good insight into the stability of attractors as the system's parameter changes. A case we will look into extensively is when we plot a \textbf{bifurcation diagram} of a one-dimensional discrete-time dynamical system, which gives a visual representation of how the attractors change as we vary a parameter. The \textbf{bifurcation points} (where the attractor type changes, doubling the period of a limit cycle each time) are surrounded by `dust' which is made by thousands of iterations taking a long time to settle around the attractor and other nearby attractors. These points have attractors that are technically fixed points or limit cycles, so should intuitively have a dimension of zero. However, the iterates converge asymptotically to this attractor so, plotting any large finite number of iterates, the attractor takes the form of a line in one-dimension; the fractal dimension of the attractor tends toward one. This tells how weakly stable these points are, leading to issues of numerical calculation around these points. On the other hand, we have \textbf{super-stable points} across the bifurcation diagram, that stay at the first iteration forever; fixed points that take no time to settle down, so have fractal dimension exactly zero. These points will be of great interest to us since numerical approximations of these points will be a key step in deriving Feigenbaum's universal constants.\\

%Varying the parameters of some system, we see changes in the types of attractor as well as fractal dimension. What this means is that there are significant parameter values in which the type of attractor changes or the fractal dimension reaches value we take interest in. Throughout this report, we look at one-dimensional discrete time dynamical systems and take great interest in the changes we see to the attractors as we alter the system's parameters. We witness a phenomenon called \textbf{bifurcation} in which a fixed point attractor becomes a period-two limit cycle, period-two becomes period-four, and so on, the period doubling each time bifurcation occurs. This change occurs at specific values of the system's parameters, called the \textbf{bifurcation points}. These points are so weakly stable that 

%An interesting case is when an object has an integer-valued fractal dimension that is higher than expected. For example, a curve in $\mathbb{R}^2$ can be so complex that it has a fractal dimension equal to two, without needing to be a space-filling curve. What this means is we can zoom into the curve at any scale and find incredible detail. These objects are important to chaos theory as they provide an intriguing and beautiful advertisement for the subject.

\chapter{Chaos with Rigour}\label{chapter:chaos_with_rigor}
\input{chaos_with_rigor.tex}

\chapter{Bifurcations in Discrete Dynamical Systems}\label{chapter:bifurcation}
The logistic map, due to its ever-repeating presence in this report, will now be formally introduced. It has become one of the staple examples of chaotic systems due to its simple form yet countless fascinating and complex properties. Furthermore, it will be the key to deriving Feigenbaum's universal constants in the final section.

\section{The Logistic Map as a Model for Population Growth}

One-dimensional recursive equations $x_{n+1} = f(x_n)$ are used for modelling various dynamical systems. 
Notwithstanding their simplicity, many of these maps exhibit extremely complicated properties which are also present in their more complicated counterparts.

Consider, for example, bacteria population in discrete time intervals. 
If the population is low and resource abundant, the rate of growth is in proportion to the population, which gives rise to the following equation
\begin{equation}\label{eq:1d iterative map}
p_{t+1} = p_{t} b,
\end{equation}
where $p_{t}$ is the population of the bacteria at the discrete time $t$ and $b$, as a constant, is the static birth rate for bacteria whose value will depend on the model.

The limited resource will slow down the rate of growth of the bacteria as the population increases, and the relation will become 
\begin{equation}\label{eq:p times beff}
	p_{t+1} =  p_{t} \be
\end{equation}

The usual assumption is that $\be$ is close to $b$ when the population is low, and there is a threshold above which the population will decrease. 
At the threshold $\be$ will be zero.

The simplest equation for modelling $\be$ is the linear equation
\begin{equation} \label{eq:b_effective}
\be = b - ap,
\end{equation}
where $a$ is another constant depending on the model.
Combining the equations \eqref{eq:p times beff} and \eqref{eq:b_effective}, the recursive formula becomes 
$$
p_{t+1}  = b p_t - ap_t^2
$$

Substituting $p_{t} := \frac{b}{a} x_{t}, \lambda := \frac{b}{4}$, we obtain
$$
x_{t+1} = 4 \lambda x_t(1-x_t) 
$$
Thence comes the standard forms of the logistic map depending on one variable $\lambda$
\footnote{
Some sources defined the logistic map as $L^*_{\lambda}(x) = \lambda x(1-x)$ without the constant 4. 
In this report, however, we will use equation \ref{eq_logistic}, which has the advantage that the maximum value attained by $\L(x)$ at $\frac{1}{2}$ is $\lambda$, and it will also be consistent to the definitions of other class of functions discussed later in the report.
}
\begin{equation}\label{eq_logistic}
	L_{\lambda}(x) = 4 \lambda x(1-x)
\end{equation}

% graph produced by `logistic_map_diff_lambda`
\begin{figure}[t]
	\centering
	\includegraphics[width=0.7\textwidth]{./figures/logistic_map_diff_lambda.png}
	\caption{Graphs of logistic map $L_{\lambda}(x) = 4 \lambda x(1-x)$ for different $\lambda$ compared with the line $y=x$ and $y = 1$.} 
	\label{fig:logistic_map_diff_lambda}
\end{figure}

% The dynamical system in discrete time interval generated by the iterations of the logistic map is the study of this session.

Scrutinising the class of discrete logistic functions plotted in figure \ref{fig:logistic_map_diff_lambda}, some of their properties are obvious:

\begin{enumerate}
	\item $\L(x)$ is a smooth function;
	\item $\L(x)$ concaves downwards, that is, $\L(x)'' < 0$;
	\item $\L(x)$ attains a unique maximum at $x = \frac{1}{2}$, and $L_{\lambda}(\frac{1}{2}) = \lambda$; and,
	\item when $0 \leq \lambda$ and $x$ is restricted to the domain $[0, 1]$, $\L(x)$ is a two-to-one non-surjective (except for $\lambda = 1$) function $\L(x): [0,1] \rightarrow [0,\lambda]$. 
\end{enumerate}


We can check if this model would work as expected by comparing it to its continuous counterpart 
\begin{equation}\label{eq_logistic_continuous}
	\frac{dp}{dx} = c p(x) (1-p(x)),
\end{equation}

where $c$ is some arbitrary constant denoting the rate of growth. 
The unique solution to this ordinary differential equation with the initial condition $p(0) = \frac{1}{2}$ is 
$$
p(x) = \frac{e^{cx}}{1+e^{cx}}.
$$
\begin{figure}
	\centering
	\includegraphics[width=0.8\textwidth]{./figures/con_vs_discrete_logistic_map.png}
	\caption{The population of the bacteria modelled by the discrete (left) v.s. continuous (right) logistic map. 
	The discrete case is modelled by $\lambda = 0.5$ and $x_0 = 0.0003$, and the continuous case has $c=1$.}
	\label{fig:con_vs_discrete}
\end{figure}
The graph of the populations modelled by \eqref{eq_logistic} and \eqref{eq_logistic_continuous} are shown in Figure \ref{fig:con_vs_discrete}.
Indeed, at least for the selected value of $\lambda$ and $c$, the population modelled by the two maps are similar.

Having settled that the discrete logistic map is a good simplification for the already-simplified equation \eqref{eq_logistic_continuous} as a model for population growth, you may wonder why we bother studying such a simple equation. 
The reason is, as simple as it seems, that the iteration of equation \eqref{eq_logistic} gives rise to some extremely complicated dynamical systems with many surprising properties. 
These interesting dynamics are also observed in the continuous case and is the study of the next session.

It shall be stressed, at the end of this session before we go in depth into the theoretical discussions, that no impression shall be made to assume that one-dimensional discrete iterative maps are only useful as a crude model for bacteria population growth. 
There are abundant examples where equation \eqref{eq:1d iterative map} can be used as a model, some up to a strikingly high accuracy. 
For example: in genetics, it is used when investigating on the frequency of the genotype \cite{genotype}; in economics, modelling the relationship between commodity quantity and price \cite{economics}; and in social science, on the propagation of rumours \cite{social_science}, among many others.


\section{Logistic Bifurcations}

To make our terminology precise and avoid any possible confusion, we shall reiterate that the focus of this session is the one-dimensional discrete dynamical system depending on one parameter $\lambda$, generated by the iterations of the logistic map. 
Explicitly, it is the sequence $x_0, x_1, \dots$, where $x_0$ is the initial condition, and $x_{n+1} = \L(x_n)$, as defined in \eqref{eq_logistic}.

The first step of studying this system is to plot it with different initial values $x$ and parameters $\lambda$. Only $0 \leq \lambda \leq 1$ and $0 \leq x_0 \leq 1$ are considered, so that $\L(x)$ is a map from $[0,1]$ to $[0,1]$. 

% graph produced by `modelling_pop_with_diff_logistic_maps` in `graph_qc` repo
\begin{figure}[htbp]
	\centering
	\includegraphics[width=0.9\textwidth]{./figures/various_iterating_logistic_map.png}
	\caption{Iterating the logistic map with different initial values and $\lambda$. All graphs are produced by setting a $x_0$ and $\lambda$, and iterating $x_{n+1} = L_{\lambda}(x_n)$, and plotting all $x_n$ values respect to iteration number $n$.}
	\label{fig:various_iter_logistic}
\end{figure}

There are several immediate observations upon looking at Figure \ref{fig:various_iter_logistic}.
When $\lambda = 0.1$ and $0.25$, it seems $\lim_{n \rightarrow \infty} x_n = 0$, regardless of the value of $x_0$.
For $\lambda = 0.6$, $\lim_{n \rightarrow \infty} x_n = l \neq 0$. 
(We can show that $l = \frac{7}{12}$ by Theorem \ref{th:_stable_unstable_fixed_point}.)

When $\lambda = 0.8$, $x_n$ no longer converges but oscillates in a stable two orbit. 
For $\lambda = 0.9$ and $\lambda = 1$, it is not clear if $x_n$ has any stable orbits or sensible patterns, and the best epithet for them would be `chaotic'.

For all cases, it seems like any initial condition, $x_0$, upon iteration, will eventually tend to some common dynamical behaviour depending only on $\lambda$.
The scatter plot in Figure \ref{fig:logistic bifurcation overview} is produced to capture the behaviour of $x_n$ as $n \rightarrow \infty$ for various intervals of $\lambda$.
These graphs are produced by picking a $\lambda$ and a random $x_0 \in [0,1]$, producing $x_{1} = \L(x_0), x_{2} = \L(x_1), \dots$ through thousands of iterations, ignoring the first several hundred $x_i$s, and plotting the rest of the $x_i$ on the coordinate $(x_i, \lambda)$. 
By repeating this process for one thousand equally spaced $\lambda \in [0,1]$, the bifurcation diagram is obtained.

These pictures of bifurcation are indeed spectacular. 
For $\lambda$ between $0$ and $a_0 = 0.25$, there is a stable fixed point $x = 0$.
When $\lambda$ becomes greater than $a_1 = 0.25$, $0$ is no longer the stable fixed point, but there is another unique stable fixed point greater than $0$.
Precisely when this stable fixed point becomes unstable, around $\lambda = a_2 \approx 0.77$, a stable two-cycle appears.
The stable two-cycle, again, disappears around $\lambda = a_3 \approx 0.86$, at which point a stable 4-cycle appears. 
This process of periodic doubling of the stable orbit, which seems to continue indefinitely, is denominated as \emph{periodic doubling bifurcation}.\footnote{
	The word bifurcation comes from medieval Latin bifurcātus, literally meaning two-forked. Here, it is used figuratively to describe the process of a single stable orbit forking into two.
	
	Bifurcations, in general, mean the change of the behaviour of dynamical systems as parameters vary.
	For systems depending only on one parameter, there are only three kinds of generic bifurcation: periodic doubling, tangent, and inverse periodic doubling \cite{Chaos_in_DS}.
	In this report, we will only touch on periodic doubling bifurcation.
	Bifurcation, in general, has very rich theory and is applicable to a large class of dynamical systems, including continuous ones and those of higher dimensions. \cite{dynamical_systems_v}.
}

Upon closer inspection of the sub-figures of Figure \ref{fig:logistic bifurcation overview}, zoomed around the windows of bifurcation, each stable orbit and bifurcation pattern seems to be self-similar, and all stable orbits cross the line $y = 0.5$ for some $\lambda$.

Periodic doubling bifurcations, however, do not exhaust the whole spectrum of $[0,1]$, and $a_{n}$, which are the values of $\lambda$ at which bifurcations occur, seem to converge to some limit as $n \rightarrow \infty$. 
This limit is labelled as $a_{\infty}$.
For $\lambda > a_{\infty}$, there are no longer any obvious patterns and the overall behaviour is best described as chaotic. 
Nevertheless, in this chaotic region there are windows for stable orbits of odd periods which are not observed for $\lambda < a_{\infty}$ (the term window describing an interval of $\lambda$ for which a stable orbit exists is introduced by May \cite{May_Nature}).
Examples are $\lambda \approx 0.957$ for a 3-cycle and $\lambda \approx 0.934$ for a 5-cycle.

Before moving to full mathematical mode and start proving, for the last time in this report we shall look back to the real world. 
As will be shown in the next session, periodic doubling bifurcation is not unique to the logistic map, but is a common behaviour for a large class of functions sharing very moderate restrictions. 
Many real world systems, such as population and density of genotype, which can be modelled by one dimensional iterative maps, exhibit dramatic variations in quantity respective to time \cite{colorado_potato_beetle}.
This behaviour has baffled biologists, many of whom have attributed the cause to the inaccuracy of measurements and perturbations from the environment. 
The fact that bifurcation is a universal phenomenon for iterative maps, however, may suggest that these variations may be due to the inherent nature of the system itself \cite{genotype}.


The following paragraph lists some of the key observations of bifurcation and the sketches of their proofs.

\begin{figure}[htbp]
	\centering
	\includegraphics[width=\textwidth]{./figures/logistic.png}
	\caption{
		The graph of logistic bifurcation.
		Each of the subfigures were produced by iterating the logistic map $\L$ with a random starting value $x_0$ to obtain $x_{n+1} = \L(x_n)$ etc., discarding the first several hundred values, and graphing each of the subsequent values as a faint, semi-transparent blue dot at the coordinate $(x, \lambda)$.
		The graph on upper left is the overview of logistic bifurcation on the interval $0 \leq \lambda \leq 1$; upper right and lower right zoomed into the area of periodic doubling, recording $0.7 \leq \lambda \leq 1$ and $0.85 \leq \lambda \leq 0.9$, respectively. Lower right is a zoomed in view of the chaotic region for $ 0.91 \leq \lambda \leq 0.97$. 
	}
	\label{fig:logistic bifurcation overview}
\end{figure}

\begin{observation}[Logistic Bifurcation]\label{th:logistic_bifurcation}
	Let $L_{\lambda} = \lambda 4x(1-x) $ be the logistic function as defined in \eqref{eq_logistic}.
	The dynamical system in a discrete time interval generated by the iteration of the logistic map has the following properties.

	\begin{enumerate}
		\item For $0 < \lambda < a_0 = \frac{1}{4}$, the system has a unique fixed point at $x = 0$. \label{log_fix_0}

		\item For $a_0 <\lambda < a_1 = \frac{3}{4}$, the fixed point at $x=0$ is no longer stable, but a new stable non-zero fixed point emerges. \label{log_fix_1}

		\item When $\lambda$ becomes greater than some value $a_2$, the one cycle becomes unstable, and a stable two cycle appears. 
		Similarly, the 2 cycle will bifurcate into a 4 cycle at $a_3$, $2^n$ cycle to $2^{n+1}$ cycle etc. until $\lambda \rightarrow a_{\infty}$. 
		\label{log_periodic_doubling}
		\item Specifically, for any $\lambda$ there is at most one set of stable orbits. \label{log_at_most_one_stable_orbit}

		\item \label{log_simul_stable_or_unstable}
		Assume for certain $\lambda$ there is an $n$ cycle. That is, there exists distinct $x_1, \cdots, x_n$ such that $\L(x_1) = \L(x_2), \L(x_2) = \L(x_3), \cdots, \L(x_n) = \L(x_1)$.
		Necessarily $x_1, \cdots, x_n$ are $n$ fixed points of $\L^n$. 
		These $n$ distinct fixed point for $\L^n$ must be simultaneously attracting fixed points or repelling fixed points for a given $\lambda$, possibly except for a set of measure zero.

		\item \label{log_cross_half} 
		Let $[a, b]$ be a window of stable orbit of period $n$.
		There exists $\epsilon \in [a, b]$ such that $L_{\epsilon}^n(0.5) = 0.5$. 
		This is the intuitive observation that every orbit must cross the line $x = 0.5$, as shown in the zoomed in view of the bifurcation in figure \ref{fig:logistic bifurcation overview}.

		\item \label{log_closest_branch}
		Observing Figure \ref{fig:logistic bifurcation overview} of the bifurcation pattern in a windows of $2^n$ stable orbits. 
		The previous observation states that there exists a certain $\lambda$ such that $0.5$ is part of the stable orbit, that is $L_{\lambda}^{2^n}(0.5) = 0.5$.
		There is another orbit closest to $0.5$, which spawned from the same bifurcation point.
		The value of this point is $\L^{2^{n-1}}(0.5)$.


		\item  \label{log_chaos_at_1}
			When $\lambda = 1$, the map exhibits chaotic behaviour, defined in Definition \ref{def:Devaney_definition_for_chaos}. 
	\end{enumerate}
\end{observation}

\begin{figure}[htbp]
	\centering
	\includegraphics[width=0.8\textwidth]{./figures/logistic_map_around_bifurcation.png}
	\caption{
		$\L, \L^2$ and the function $y=x$ are graphed and zoomed in around the fixed point for $\lambda = 0.75$ and $\lambda = 0.78$, where the bifurcation takes place.
		For $\lambda < 0.75$, as can be seen from Figure \ref{fig:logistic_map_diff_lambda}, the system as a unique stable point. This point loses its stability and a stable two-cycle appears just after $\lambda$ increases above $0.75$.
		This is because, as shown on the right, precisely when the stable fixed point became unstable, the graph of $\L^2$ will have two more intersection with the line $y=x$, which becomes the stable two orbit.
	}
	\label{fig:point_of_bifurcation1}
\end{figure}

\begin{figure}[htbp]
	\centering
	\includegraphics[width=0.8\textwidth]{./figures/logistic_map_around_bifurcation_2.png}
	\caption{Showing $\L^2, \L^4$ and $y=x$ close to one point of second bifurcation similar to Figure \ref{fig:point_of_bifurcation1}}.
	\label{fig:point_of_bifurcation2}
\end{figure}

\begin{proof}[Proof of \ref{th:logistic_bifurcation}.\ref{log_fix_0} and \ref{th:logistic_bifurcation}.\ref{log_fix_1}]
	The fixed point for $\L(x)$ is exactly the solution of the equation $\L(x) = x$, which are $x_0 = 0$ and $x_1 = 1 - \frac{1}{4\lambda}$. 
	$|L_{\lambda}(x_0) | = 4 \lambda < 1$ for $0 < \lambda < \frac{1}{4}$, so by Theorem \ref{th:_stable_unstable_fixed_point}, $x_0$ is a stable fixed point in this interval.
	The argument for $x_1$ is similar.
\end{proof}

% TODO: This prove needs improvements
\begin{proof}[Demonstration of \ref{th:logistic_bifurcation}.\ref{log_periodic_doubling}]
	The proof of this phenomenon again exploits Theorem \ref{th:_stable_unstable_fixed_point}.

	Let us concentrate on $\lambda^*$ at which the fixed point $a$ becomes unstable; that is, $L_{\lambda*}' a  = - 1$ and for any $\lambda > \lambda^*$, the derivative at the fixed point is smaller than $-1$, thus $a$ becomes an unstable fixed point. (This observation is made obvious in Figure \ref{fig:point_of_bifurcation1}.)
	Necessarily, $\frac{d}{dx}L_{\lambda^*}^2 a = L'_{\lambda^*}(a) \cdot L'_{\lambda^*}(a) = 1$,
	and for any $\lambda > \lambda^*$, $\frac{d}{dx}L_{\lambda}^2a'> 1$, where $a'$ is the new fixed point.

	Since $\frac{d}{dx}L_{\lambda}^2(a') > 1$, by mean value theorem for some $\delta > 0$, $\L^2(a' - \delta) < a' - \delta$, and $\L^2(a' + \delta) > a' + \delta$. 
	Observe that $\L^2(1) = 0 < 1$, so by intermediate value theorem there must be some point $b > a' + \delta$ such that $\L^2(b) = b$. 
	Around $a'$ $L^2$ is concaving upwards, meaning its derivative is decreasing, this means that $\frac{d}{dx }L_{\lambda}^2(b) < 1$,
	and we can pick some $\lambda$  close to $\lambda ^*$ such that $-1<\frac{d}{dx }L_{\lambda}^2(b) < 1$, and at this point $b$ is a stable fixed point of $L_{\lambda}^2$.
	The argument for the other fixed point smaller than $a'$ is similar. 

	One significant observation is that the flip of sign of the derivative at the point of bifurcations. 
	To put it precise, at $0.75$ when the fixed point $a$ of $\L^1$ about to lose its stabilitym $\frac{d}{dx} \L(a) = - 1$, and the new fixed point $a'$ of $\L^2$ has $\frac{d}{dx} \L^2(a^*) = 1$. 
	As $\lambda$ increases, the derivative of $\L^2$ at $a^*$ will decrease, and at $a_2$ it will surpass $-1$ and become unstable, and the derivative of the new fixed point of $\L^4$ will be $1$, which will again decrease as $\lambda$ increases.
	This process continues indefinitely.

	All of our arguments above are qualitative, involving only the signs of the derivative and second derivative. 
	Restricting our attention to a neighbourhood of $L^2$ around the point of second bifurcations, all of the above arguments can be applied to $L^2$ and $L^4$. 
	This is the intuition why the bifurcation would continue indefinitely.
	
	The above statement can be made precise and rigourous by the use of Schwarzian derivative, as shown in chapter 12 of \cite{Devaney_green_book_chaos_definition}.
	\footnote{
		The Schwarzian derivative of a function $f$ as $x$ is 
		$$
		Sf(x) = \frac{f'''(x)}{f'(x)} - \frac{3}{2} \left(\frac{f''(x)}{f'(x)}\right)^2.
		$$
		The details derivations of Schwarzian derivative and its application is beyond the scope of this report. 
		Here some of the most fundamental properties are listed.
		\begin{enumerate}
			\item A polynomial $p(x)$ all of whose roots are real and distinct has negative Schwarzian derivative.
			\item If $Sf < 0, Sg < 0$, then $S(f\circ g) < 0$.
			\item If $Sf <0$, $f$ can not have a positive local minimum or negative local maximum.
			\item If $Sf < 0$ and $f$ has finitely many critical points, then $f$ has finitely many attracting points of period $n$ for each $n$.
		\end{enumerate}
	}
\end{proof}

\begin{proof}[Demonstration of \ref{th:logistic_bifurcation}.\ref{log_at_most_one_stable_orbit}]
	This follows from the previous point, but also stems from the fact that $\L^n$ is a function of negative Schwarzian derivative with bounded interval for stable points.
	The number of stable orbits of such maps must not exceed the number of critical points.

	This fact is proved in \cite{Pierre_Collet} and in chapter 11 of \cite{Devaney_green_book_chaos_definition}.
\end{proof}


\begin{proof}[Proof of \ref{th:logistic_bifurcation}.\ref{log_simul_stable_or_unstable}]
		Assuming there exists distinct $x_1, \cdots, x_n$ such that $\L(x_1) = \L(x_2), \L(x_2) = \L(x_3), \cdots $, and $\L(x_n) = \L(x_1)$.

		Differentiate $\L^n$ and evalute at $x_1$ 
		$$
		\frac{d}{dx} \L^n(x_1) = \prod_{i=1}^n \L'(x_i)
		$$

		Indeed evaluting at any other $x_i$ gives the same value. 
		So except for a set of $\lambda$ such that $\frac{d}{dx} \L^n(x_1) = \pm 1$, these $n$ points regarding as fixed points of $\L^n$ must be simultaneously attracting or repelling fixed points by Theorem \ref{th:_stable_unstable_fixed_point}.
\end{proof}

\begin{proof}[Proof of \ref{th:logistic_bifurcation}.\ref{log_cross_half}]
	As shown in the proof of \ref{th:logistic_bifurcation}.\ref{log_simul_stable_or_unstable}, for an $2^n$ cycle of $x_1, x_2, \cdots, x_{2^n}$, differentiate $\L^{2^n}$ and evaluate at any of $x_i$ gives the same value
	$$
	\frac{d}{dx} \L^{2^n}(x) = \prod_{i=1}^{2^n} \L'(x_i)
	$$
	
	In the proof of \ref{th:logistic_bifurcation}.\ref{log_periodic_doubling} we have shown that in a window of stable $2^n$ orbit, the $\frac{d}{dx} \L^{2^n}$ decreases from $1$ to $-1$ as $\lambda$ increase.
	We may assume the derivative at the fixed point is a continuous function of $\lambda$, so by intermediate value theorem there must be some $\lambda$ such that the derivative is $0$. 
	As the derivative equals to the finite product $\prod_{i=1}^{2^n} \L'(x_i)$, one of the $\L'(x_i)$ must be $0$.
	Logistic map has a unique maximum $0.5$, and this will take place iff one of the $x_i = 0.5$.
\end{proof}

\begin{proof}[Demonstration of \ref{th:logistic_bifurcation}.\ref{log_closest_branch}]
	Assuming $a_0$ is one of the points in a stable $2^{n-1}$ orbits, then $\L^{2^{n-1}}(a_0)=a_0$ for $\lambda$ in its window.
	As $\L(x)$ is a continuous function of both $x$ and $\lambda$, small perturbations of $\lambda$ will result in small perturbations of $\L^{2^{n-1}}(a_0)$, so, after $\lambda$ increase beyond the window for $2^{n-1}$ orbit and $\L^{2^{n-1}}(a_0) \neq a_0$, but its value shall still stay close to $a_0$.
\end{proof}


\begin{proof}[Proof of \ref{th:logistic_bifurcation}.\ref{log_chaos_at_1}]
	The doubling map \eqref{eq:doubling_map}, which is chaotic, is topologically conjugate to $L_{\lambda}$, where $\lambda = 1$, so $L_{\lambda}$ is chaotic.
	This is proved in Example \ref{ex_logistic_and_doubling}.
\end{proof}

\section{Feigenbaum's Constants}

Another striking observation from Figure \ref{fig:logistic bifurcation overview} is that the overall shape of the graph exhibits some kind of fractal structure that is \emph{self similar}.
If only focusing on one branch of bifurcation and disregarding the coordinates, the shape of each bifurcation is similar to any other up to elongation and stretching, including the branches bifurcated out from itself.
 
 This observation is crucial to many properties of iterated maps and leads naturally to the conjecture that each bifurcation is scaled down from its parent, and there are some constants to describe the ratio of the scaling. 

 There are two ways to quantify this self-similarity: the spacing between each bifurcation points, and the distance between the superstability point and the point closest to it in the stable orbit. 
 In either case we can compute numerically the values of $x$ and $\lambda$.

 Numerical computation of the bifurcation point beyond the first few terms is difficult, as near the bifurcation points the iterated maps converge very slowly.
 Instead, we can compute the point of superstability where the rate of the convergence is fastest by using Theorem \ref{th:logistic_bifurcation}.\ref{log_cross_half}, which states that each of the stable orbits must cross $0.5$ where they achieve superstability ($\frac{d}{dx}\L^{2^n}(0.5) = 0$).
In this report we will label the value of $\lambda$ as $A_n$ at which the superstability point of $2^n$ cycle is attained.
Once $A_n$ are known, Theorem \ref{th:logistic_bifurcation}.\ref{log_closest_branch} states that the coordinate of the point in the bifurcation cycle closest to $0.5$ (which is also part of the bifurcation cycle as super-stability is attained) is $\L^{2^{n-1}}(0.5)$. (Recall at $A_n$ there a stable $2^n$ cycle emerges.)
Therefore the distances between them is $|\L^{2^{n-1}}(0.5) - 0.5|$ and are labelled as $d_n$.
Figure \ref{fig:demonstration of feigenbaum constants on logistic map} is a demonstration of $A_i$ and $d_i$ on logistic map.

The values of $A_n$ and $d_n$ calculated numerically are shown in Table \ref{tab:feigenbuam_alpha_table_for_logistic}. 
The observation is clear:
$\frac{A_{n+1}-A_n}{A_{n+2}-A_{n+1}}$ approaches approximately $4.668$ as $n \rightarrow \infty$, and $\frac{d_n}{d_{n+1}}$ approaches approximately $2.503$ as $n \rightarrow \infty$. 
The former is known as Feigenbaum's constant $\delta \approx 4.6692016091023$, the latter Feigenbaum's constant $\alpha \approx 2.5029078750957$ \cite{F1}.
The reason these constants are worth such as denominations is that they are \emph{universal}.

\begin{table}
\centering
\begin{tabular}{|c|c|c|c|c|}
\hline
\( n \) & \( A_n \) & \( d_n \)  & \(\frac{A_{n+1} - A_n}{A_{n+2} - A_{n+1}}\)  &  \(\frac{d_n}{d_{n+1}}\) \\ \hline
0 & 0.5000000000 & - & 4.5358092997 & - \\
1 & 0.8064950000 & 0.3064950000 & 4.6838460230 & 2.6616365278 \\
2 & 0.8740672853 & 0.1151528380 & 4.6103617168 & 2.5423697165 \\
3 & 0.8884939520 & 0.0452935060 & 4.6110890246 & 2.5175763063 \\
4 & 0.8916231352 & 0.0179909169 & 4.6981361800 & 2.5029942863 \\
5 & 0.8923017565 & 0.0071877579 & 4.5692030565 & 2.5187506567 \\
6 & 0.8924462013 & 0.0028536996 & 4.6976886127 & 2.5019617749 \\
7 & 0.8924778140 & 0.0011405848 & 4.5707037850 & 2.5191930664 \\
8 & 0.8924845434 & 0.0004527580 & 4.6690000000 & 2.5048153706 \\
9 & 0.8924860157 & 0.0001807550 & 4.6410832897 & 2.5116759261 \\
10 & 0.8924863310 & 0.0000719659 & 4.5599414736 & 2.5226365150 \\
11 & 0.8924863990 & 0.0000285281 & 4.6418936559 & 2.5096745464 \\
12 & 0.8924864139 & 0.0000113672 & 4.6286915064 & 2.5130534587 \\
13 & 0.8924864171 & 0.0000045233 & 4.6027721102 & 2.5181104605 \\
14 & 0.8924864178 & 0.0000017963 &  - & 2.5201147995 \\
15 & 0.8924864179 & 0.0000007128 &  - &  - \\
\hline
\end{tabular}
\caption{
	Values of \( A_n \), \( \delta_n \), and their ratios for the logistic map.
	Each $a_n$ is the value of $\lambda$ such that $\L^{2^n}(0.5) = 0.5$, which is the value of $\lambda$ where the $2^n$ cycle crossed the $y=0.5$ line. 
	Each $\delta_n$ is the difference between between 0.5 and the closest point in the $2^n$ cycle at $A_n$.
	The ratios of there difference are also calculated.
}
\label{tab:feigenbuam_alpha_table_for_logistic}
\end{table}

\begin{figure}
	\centering
	\includegraphics[width=0.8\textwidth]{./figures/demonstration of feigenbaum constants.png}
	\caption{ 
		A portion of the bifurcation diagram of the logistic map with demonstration of  $A_i$, which is the value of $\lambda$ at which the superstability of $2^i$ cycle is attained, and $d_i$, which is the distance between the superstability point (at $x=0.5$) and the closest point in the stable orbit.
		The $\lambda$ axis in in logarithmic scale, and the transformation to from $\lambda$ to $x$ coordinate is $-\log(A_{\infty} - \lambda)$.
	}
	\label{fig:demonstration of feigenbaum constants on logistic map}
\end{figure}

It is worthwhile to describe our algorithm to compute $A_i$ and $d_i$ in detail, as such seemingly innocuous computation is actually surprisingly challenging.
As for as the authors' knowledge, computations of $A_i$ to such precision has not been performed previously, and the method described here is original.

The first challenge is the precision of floating point arithmetic.
At $n = 10$, the value of $A_{n+1} -A_{n} $ is smaller than $ 10^{-8}$. 
To compute $\frac{A_{n+1} - A_n}{A_{n+2} - A_{n+1}}$ to $8$ significant figures (base 10), therefore, would require calculating $A_i$ to 16 significant figures, greater than the precision of the double precision floating point used in most computers, which only allows approximately 15 significant figures
\footnote{
	The IEEE standard for floating point arithmetic \cite{IEEE_floating_point} (IEEE 754) decrees 53 bits of binary digits in double precision floating point (page 8, table 3.2 of the standard.) This translates to $53 \cdot \log_{10}(2) \approx 15.8$ significant digits based 10.
}.
The only way to overcome this is to use an arbitrary precision arithmetic library. 
We chose to use GNU GMP (GNU Multiple Precision Arithmetic Library) \cite{GMP} with interface in C++.

The next challenge is the efficiency of the algorithm. 
If the algorithm starts by trying each $\lambda = 0$ and increments by $10^{-8}$ until $\lambda$ reaches $1$, the computer needs to compute literally billions of iterations of the logistic map, taking very long time and much wasted computer power.
The natural optimisation is to guess approximately where $A_i$ shall be and check $\lambda$ with greater precision in that area. 
In the process of guessing we still need to increment $\lambda$ by some difference, but initial increment of $\lambda$ can be relatively large, and whenever a value of $A_i$ is found, this difference shall be scaled down by Feigenbaum's  $\delta$.


While the full code in C++ was included in the appendix,
Algorithm \ref{ag:compute A_i} shows the pseudocode, which needs input parameters $\lambda_{start}$ for the start of $\lambda$, $\lambda_d$ for the initial increment of $\lambda$, and $n_{bound}$ for the maximum number of $A_i$ to be computed.
Algorithm \ref{ag:compute A_i} depends on Algorithms \ref{ag:super stable} and \ref{ag:fine tuning}. 
The former checks if superstability under a threshold is obtained at a certain $\lambda$, and the latter fine tunes $\lambda$ to obtain the most accurate value of $A_i$.
Both algorithms require various parameters including the threshold for superstability and the constant $c$ for fine tuning.
By poking different values for parameters of this algorithm, reasonably accurate numerics can be obtained up to $A_{15}$.


\begin{algorithm}
	\caption{Check if $\lambda$ is super stable}
	\begin{algorithmic}[1]
		\Require $\lambda$, cycle number $n,$ point of superstability $x^* = 0.5$, and threshold $t$.
		\If{$|L_{\lambda}^{2^{n-1}}(0.5) - 0.5| < t$}
		\State \Return True
		\Else
		\State \Return False
		\EndIf
	\end{algorithmic}
	\label{ag:super stable}
\end{algorithm}

\begin{algorithm}
	\caption{Fine Tuning $\lambda$}
	\begin{algorithmic}[1]
		\Require $\lambda$, cycle number $n,$ point of superstability $x^* = 0.5$, and an constant $c$. $\lambda \in (\lambda - c, \lambda +c)$ are checked.
		\State Create a list $L$ of $\lambda$ in the interval $I$.
		\State Compute the list $L^*$, which holds $L_{\lambda}^{2^{n-1}}(0.5)$ for each $\lambda$ in $L$.
		\State Set $\lambda$ to be the element in $L$ corresponding to the minimum value in $L^*$.
		\State Perform fine tuning by repeating the above steps after setting $c = c / \delta$. \Comment{$\delta$ is Feigenbaum's delta}
		\State Iteration stops when $c$ is smaller than a certain threshold.
	\end{algorithmic}
	\label{ag:fine tuning}
\end{algorithm}

\begin{algorithm}
	\caption{Computation of $A_i, d_i$}
	\begin{algorithmic}[1]
		\Require $\lambda = \lambda_{start}$, $A_i = \{0.5\}, d_i = \{\}$, and cycle number $n = 2$ (We have computed the first stability point is $0.5$, so we can start with cycle number $2$.)
		\While{$\lambda < \lambda_{end}$ and $len(A_i) < n_{bound} $}
		\If{$L_{\lambda}(0.5)$ is super stable}  \Comment{Using algorithm \ref{ag:super stable}}
			\State Fine tuning $\lambda$ with algorithm \ref{ag:fine tuning}.
			\State Append $\lambda$ to $A_i$ 
			\State Append $\L^{2^{n-1}}(0.5)$ to $d_i$ 
			\State $\lambda = \lambda + \lambda_{i}$
			\State $\lambda_{d} = \lambda_d / \delta$  \Comment{$\delta$ is Feiganbaum's delta}
			\State $\lambda_{i} = \lambda_i / \delta$
			\State $n = n + 1$
		\EndIf
		\State $\lambda = \lambda + \lambda_d$
		\EndWhile
	\end{algorithmic}
	\label{ag:compute A_i}
\end{algorithm}

\section{Single Nodal Functions Give Rise to Bifurcations}

Our previous discussion of infinite bifurcations of the logistic map is qualitative and the proof of Theorem \ref{th:logistic_bifurcation} only relies on the facts that the logistic map is differentiable, concaves downwards, has a unique maximum, and has negative Schwarzian derivative. 
It may not be of surprise, therefore, that the dynamical properties of bifurcation are shared among a large class of functions with very moderate restrictions.
Let us investigate some more examples.

Figure \ref{fig:combined_bifurcations} shows the bifurcation diagrams of $f(\lambda, x) = \lambda \sin(2\pi x)$, $f(\lambda, x) = \lambda + \sin(2\pi x)$, $f(\lambda, x) = \lambda x(1-x)^2$, and $f(\lambda, x) = \lambda x \log(x)$.
Infinite bifurcation and periodic doubling to chaos were observed for all cases, although the exact values of $\lambda$ at which the bifurcation take place is different for each map.
It is also striking that the shape of each bifurcation diagram is extremely similar to that of the logistic map up to elongation and stretching, so it is reasonable to conjecture that most of the points of Theorem \ref{th:logistic_bifurcation} shall also apply to these functions.
Indeed, point \ref{log_cross_half} of Theorem \ref{th:logistic_bifurcation} is readily demonstrated by Figure \ref{fig:combined_bifurcations}
as, in each subfigure, the horizontal line at the local maximum crosses the stable orbits of any periods at some $\lambda$.

% TODO: How shall we name these diagram if there is no bifurcation?
Figure \ref{fig:combined_no_bifurcations} shows the bifurcation (or the lack of bifurcation) diagrams of $\lambda \sin(2\pi x)$, $\lambda + \sin(2\pi x)$, $\lambda x(1-x)^2$, and $\lambda x \log(x)$. 
Each of these functions exhibit dynamical behaviour different from periodic bifurcations.
The $\lambda e^x$ map bifurcated precisely once at $-e$. 
For all $\lambda < -e$ there are two stable orbit, while for $\lambda > -e$ there are one stable orbits.
As for the tent map, there is one unique stable fixed point $x=0$ for $\lambda \in [-0.5, 0.5]$, and for any other $\lambda$ there seems to be no stable orbits at all but homogeneous region of chaos. 
The fact that tent map is chaotic is proved in Example \ref{ex:logistic and tent}.
The two $\arctan$ maps have similar behaviours to the exponential map. 
Their stable orbit only bifurcated once over all $\lambda \in \bb{R}$.

Why do the former class of functions exhibit infinite bifurcations to chaos, but the latter class do not?
The graphs of these functions, shown in Figure \ref{fig:combined_bifurcations_functions_graph}, may give the answer. 
All functions which give rise to bifurcations are smooth and have single-modal structures; that is, their values are bounded and attain unique maximums in some interval. Those which do not exhibit bifurcation are either not differentiable at the maximum, e.g. the tent map, or do not have local maxima, e.g. the exponential and arctangent maps.
This observation fits our expectation, as in the proof of Theorem \ref{th:logistic_bifurcation}, the most important property used is the fact that the logistic function attains a unique, differentiable maximum at $x=0.5$ and it concaves downwards. 
Metropolis et al. \cite{metropolis2017finite} have shown that the bifurcation phenomenon is universal, as summarised in the following theorem.

\begin{thm}[Criterion for infinite bifurcations]\label{th:criteria_for_infinite_bifurcations}
	If $f(x)$ satisfies the following properties:
	\begin{enumerate}
		\item $f(x)$ is continuous, singled valued, piecewise $C^1$ and has a unique differentiable maximum on the interval $[0,1]$ attained at $x^*$;
		\item $f(x) > 0$ on $(0,1)$, $f(0) = f(1) = 0$, and $f$ is strictly increasing in $[0, x^*]$ and strictly decreasing in $[x^*, 1]$;
		\item for $\Lambda_0 < \lambda < 1, \lambda f(x)$ has two fixed points (one of which is $x = 0$) both of which are repellent, that is $|\lambda f'(x)| > \frac{1}{\lambda}$; and,
		\item in the interval $N$ around the unique maximum $f$ concaves down-wards,
	\end{enumerate}
	then the system $x_{n+1} = \lambda f(x_n)$ will exhibit infinite bifurcations and periodic doubling to chaos.
	To be precise, the criterion for chaos by period doubling bifurcations is:
	\begin{enumerate}
		\item for $\Lambda_0 < \lambda < \lambda_{\infty} < 1$, there exists stable orbit of period $2^n$ for all $n = 1, 2, \cdots$. with $n$ increasing with $\lambda$; and
		\item for each $\lambda$, there exists at most one set of stable orbits.
	\end{enumerate}
\end{thm}

\begin{figure}
	\centering
	\includegraphics[width=\textwidth]{./figures/combined_bifurcations.png}
	\caption{
		Bifurcations diagrams of four different functions 
		$ \lambda \sin(2\pi x)$,
		$ \lambda + \sin(3\pi x)$,
		$ \lambda x(1-x)^2$,
		and $ \lambda x \log(x)$.
		All of these functions exhibit bifurcations and periodic doubling to chaos like the logistic map. 
		The subfigures on the left are the overview of bifurcations, where the ones on the right are zoomed in around points of bifurcation.
		The graph of these functions are presented in Figure \ref{fig:combined_bifurcations_functions_graph}
	}
	\label{fig:combined_bifurcations}
\end{figure}

\begin{figure}
	\centering
	\includegraphics[width=\textwidth]{./figures/combined_no_bifurcations.png}
	\caption{
		These figures show patterns of (or lack of) stable orbits for four functions which do not exhibit periodic doubling to chaos. 
		The graph of these functions were presented in Figure \ref{fig:combined_bifurcations_functions_graph}.
	}
	\label{fig:combined_no_bifurcations}
\end{figure}

\begin{figure}
	\centering
	\includegraphics[width=\textwidth]{./figures/combined_functions.png}
	\caption{
		This figure shows the graph of the functions whose bifurcation diagrams are graphed on Figure \ref{fig:combined_bifurcations} and \ref{fig:combined_no_bifurcations}.
	}
	\label{fig:combined_bifurcations_functions_graph}
\end{figure}


Having settled that the periodic bifurcation is a universal phenomenon for a large class of functions, naturally we ask how about the scaling constants shown in Table \ref{tab:feigenbuam_alpha_table_for_logistic}.
The surprising fact is that Feigenbaum's constants $\delta$ and $\alpha$ are universal for any iterative map depending on one variable and exhibit periodic doubling bifurcations with orbit $2^0, 2^1, 2^2, \cdots$. 
Indeed, Tabor showed that such constants are universal for iterative maps with negative Schwarzian derivative \cite{Tabor}.
The entirety of the last chapter will be dedicated to the demonstration of this fact based on Feigenbaum's theory of scaling. 
Here, instead, we present another numerical evidence for the universality of Feigenbaum's constants.

Table \ref{tab:feigenbuam_constants_skewed_logistic_map} shows the $A_i$, $d_i$, and their ratios for the skewed logistic map $f_{\lambda} (x) = \lambda x (1-x)^2$.
The numerics were calculated with an algorithm similar to Algorithm \ref{ag:compute A_i}.
Although the speed of convergence is clearly different, 
$ \frac{d_n}{d_{n+1}} $ and $\frac{A_{n+1} - A_n}{A_{n+2} - A_{n+1}}$ seem to converge to the same number as in the logistic map.
\begin{table}
\centering
\begin{tabular}{|c|c|c|c|c|}
\hline
\( n \) & \( A_n \) & \( d_n \)  & \(\frac{A_{n+1} - A_n}{A_{n+2} - A_{n+1}}\)  &  \(\frac{d_n}{d_{n+1}}\) \\ \hline
0 & 2.25000000 & - & 3.58291025 & - \\
1 & 4.49325749 & 0.33233444 & 4.41239995 & 3.11334650\\
2 & 5.11935677 & 0.10674508 & 4.60351732 & 2.31871272 \\
3 & 5.26125217 & 0.04603635 & 4.65268653 & 2.58607785 \\
4 & 5.29207543 & 0.01780161 & 4.66307535 & 2.47302222 \\
5 & 5.29870026 & 0.00719832 & 4.65793977 & 2.51729830 \\
6 & 5.30012096 & 0.00285954 & 4.67119534 & 2.49830019 \\
7 & 5.30042596 & 0.00114459 & 4.64986027 & 2.50813287 \\
8 & 5.30049126 & 0.00045635 & 4.65947170 & 2.50369628 \\
9 & 5.30050530 & 0.00018227 & 4.66753118 & 2.50386731 \\
10 & 5.30050832 & 0.00007279 & 4.66237768 & 2.50460649 \\
11 & 5.30050896 & 0.00002906 & 4.64855560 & 2.50672818 \\
12 & 5.30050910 & 0.00001159 & 4.65083889 & 2.50600317 \\
13 & 5.30050913 & 0.00000462 & 4.66534446 & 2.50451859 \\
14 & 5.30050914 & 0.00000184 &  - & 2.50767228   \\
15 & 5.30050914 &  0.00000073&  - &  - \\
\hline
\end{tabular}
\caption{
	Values of \( A_n \), \( \delta_n \), and their ratios up to ten decimal places for the skewed logistic map $f_{\lambda}(x) = \lambda x(1-x)^2$.
	The calculations are the same as in Table \ref{tab:feigenbuam_alpha_table_for_logistic}.
}
\label{tab:feigenbuam_constants_skewed_logistic_map}
\end{table}

\chapter{Feiganbaum's Theory of Scaling}
\section{Feigenbaum's Theory of Scaling}

The whole theory of bifurcation is spectacular; even more amazing are the Feigenbaum constants $\alpha$ and $\delta$, whose existence and universality are demonstrated by the numerics in Tables \ref{tab:feigenbuam_alpha_table_for_logistic} and \ref{tab:feigenbuam_constants_skewed_logistic_map}. 
The existence of universal constants across a wide class of chaotic dynamical systems is a quintessential quantifiable property of chaos.
The proof that these constants exist and are universal is well beyond the scope of this undergraduate project, but we can give some rational argument as to the existence of $\alpha$.

\begin{figure}
    \centering
    \includegraphics[width=0.6\linewidth]{Images/demonstration of feigenbaum constants.png}
    \caption{Bifurcation diagram of the logistic map, which shows the location of four parameter values $A_n$ which correspond to where each $2^n$ cycle achieves superstability. The distances $d_{n+1}$ are those between $X_m$ and the closest point in the cycle.}
	\label{fig:universal_raito}
\end{figure} 

For a given iterative map $F$ depending on one parameter satisfying the requirement of Theorem \ref{th:criteria_for_infinite_bifurcations}, $d_i$ is the distance at the superstability of the $2^{i}$ cycle between $X_{m}$ and its closest point in the cycle, where $X_m$ is the unique maximum of $F$.

Recall that $\alpha = \lim_{i \rightarrow  \infty}\frac{b_i}{b_{i-1}}$. 
Writing it in another way
\begin{equation}\label{eq:d_sim}
d_r \sim\frac{D}{(-\alpha)^r} \quad \quad \text{ as } r \to \infty
\end{equation}
where $D$ is a constant depending on $F$ and $\alpha$ is Feigenbaum's $x$-scale constant. 

Theorem \ref{th:logistic_bifurcation}.\ref{log_closest_branch} states that $d_i = F^{2^{n-1}}(A_i, X_m) - X_m$, where $A_i$ is the value of the parameter at which the $2^n$ cycle achieves superstability and crosses $X_m$. An example of these distances is shown in Figure \ref{fig:universal_raito}.
Therefore, to find the scaling of $d_r$, we take the limit of \eqref{eq:d_sim},

\begin{equation}
\lim_{r \to \infty} (-\alpha)^r \left(F^{2^r}(A_{r+1}, X_m) - X_m\right) = - D/\alpha.
\end{equation}

This leads to the hypothesis that
$$
g_1(x-X_m)=\lim_{r \to \infty} \{g_{1r}(x-X_m)\},
$$
where
$$
g_{1r}(x-X_m) = (-\alpha)^r \{F^{2r}(A_{r+1}, X_m + (x-X_m)/(-\alpha)^r)-X_m\}.
$$
Let $x-X_m$ be the difference from the point of superstability about the bifurcation to $X_m$. 
From the scaling of $d_r$, this implies that $g_1(0)=-\frac{D}{\alpha}$ and by solving for the first couple of iterations of $g_{1r}(x-X_m)$, we can see a limit emerges for $g_1$.
Feigenbaum believed that $g_1$ is dependent on the scaling of $x-X_m$ which shows the behaviour of $F$ near to its maximum. 
% Since $g_1$ doesn't matter on the map $F$ we can thus say that this behaviour causes $g_1$ to become universal.

Without loss of any generality we can set $X_m=0$.
The main ideas of scaling $x$ can be seen by the operator $T$, which we define to be
\begin{equation} \label{eq: operator T}
T\psi(x)=-\alpha \psi (\psi(-\frac{x}{\alpha})).
\end{equation}
Setting $\psi=g_1$, we have
\begin{align}
    Tg_1 &=-\alpha g_1(g_1(-x/\alpha)) \nonumber \\
    &= -\alpha \lim_{r \to \infty} \{(-\alpha)^rF^{2r}(A_{r+1}, (-\alpha)^rF^{2r}(A_{r+1},x/(-\alpha)^{r+1}))\}.  \label{eq:one}
\end{align}
Now, if we set $\phi(y)=(-\alpha)^rF^{2r}(A_{r+1},y/(-\alpha)^r))\}$ such that $$\phi^2=\phi(\phi(y))=(-\alpha)^rF^{2r+1}(A_{r+1},y/(-\alpha)^r))\}$$ and $y=\frac{x}{\alpha}$, we can see that
\begin{align*}
    Tg_1 &= \lim_{r \to \infty} \{(-\alpha)^{r+1}F^{2r+1}(A_{r+1}, x/(-\alpha)^{r+1})\} \\
    &= \lim_{q \to \infty} \{(-\alpha)^qF^{2q}(A_q,x/(-\alpha)^q)\} \\
    &= g_0(x)
\end{align*}
This then implies that $Tg_k(x)=g_{k-1}(x)$ for $k = 2,3,...$ where we have shown that,
\begin{align}
    g_k=\lim_{r \to \infty} \{(-\alpha)^{r}F^{2r}(A_{r+k}, x/(-\alpha)^{r})\}. \label{eq:two}
\end{align} 
We can now take the limit as $k \to \infty$ to show that there exists a function $g(x)= \lim_{k \to \infty}g_k(x)$ such that  the functional equation
\begin{align}
    Tg=g \label{eq:FunctionalOperator}
\end{align}
holds. 
And $g$ is a `fixed point' of the nonlinear functional operator $T$.

Using the functional equation \eqref{eq:FunctionalOperator}, the constant $\alpha$ must be determined as part of the solution. 
The function $g(x)$ satisfies this functional equation, and it possesses a scaling symmetry, such that if $g(x)$ is a solution, then the rescaled function $\mu g(x/\mu)$ is also a solution for any $\mu \neq 0$. 
This scaling symmetry implies that there is an inherent freedom in the choice of the parameter $\mu$.
By convention, $\mu$ is chosen such that 
$$
g(0)=1.
$$
Setting $x=0$ in the functional equation, we have, by the original definition  of our operator,
\begin{equation}\label{eq:alpha}
\alpha = -1/g(1).
\end{equation}

Feigenbaum was able to numerically verify the method, where he found that
\begin{align}
    g(x)= 1 - 1.52763x^2 + 0.10482x^4- 0.02671x^6 + \dots .\label{eq:feigenbaum}
\end{align}
This corresponds to the desired $\alpha =-1/g(1)=2.5029$.

\begin{exmp} The Logistic Map
	The logistic map, defined by $F(a,x)=ax(1-x)$, attains its maximum value at $X_m =1/2$.
	Thus we are able to replace $x-X_m$ with $x-1/2$. Hence our new function is given as $F(a,x)=a(1/4-x^2)-1/2$, ensuring that the maximum value of our new function occurs at $x=0$. Using equations \eqref{eq:one} and \eqref{eq:two} we can now see that,
    \begin{align}
        g_{10}&=F(A_1,x)=A_1(1/4-x^2)-1/2 \nonumber \\
        g_{11}&=(-\alpha)F^2(A_2,x/(-\alpha)) \nonumber \\
        &= \alpha A_2 \left(\left(A_2 \left(x^{2}/\alpha^2 - 1/4\right) + 1/2\right)^{2} - 1/4\right) + \alpha/2
    \end{align}
    Thus, for $g_{1r}(x)$, we can plot our findings as $r \to \infty$ in the $x, y$ plane where $y=g_{1r}(x)$. 
	When we present our first couple of plots, we can see a clear limiting function $g_{k=1}$ appearing as $r$ increases. Thus from \eqref{eq:two} we can see that, as we increase $k$, the sequence of functions converges to some function $g(x)$. 
	This is demonstrated in Figure \ref{fig:unscaled}.
	\begin{figure}
    \centering    \includegraphics[width=1\textwidth]{Feigenbaum Approx Graphs/feignabum.png}
    \caption{Unscaled graphs corresponding to the logistic map showing the convergence to a limiting function as we iterate over $r,k$. The lines don't converge to Feigenbaum's results meaning that there must exist a scaling factor $\mu$ which needs to be applied.}
    \label{fig:unscaled}
\end{figure}
    

    The blue line in Figure \ref{fig:unscaled} shows us Feigenbaum's numerically solved solution \eqref{eq:feigenbaum} while the rest of the lines show us the iterations of $r$ for specified $k$ values. 
	By iterating $r$ the function seems to converge to some limit, which, however, is different from the solution of Feigenbuam.
	This issue can be resolved by rescaling the function $g(x)$ by an appropriate factor $\mu = \frac{1}{g(0)}$, where $g(0)$ here is our map's solved unscaled value at 0. 
    From our findings, we found that our value of $g(0)$ is $0.39176186028018406$ such that the corresponding scaling factor was found to be $\mu= 2.552571093277968$. Now that we have our new rescaled function $g(x)$, we can plot the iterations of $r$ and $k$. 
	The results are shown in Figure \ref{fig:rescaled}.
    \begin{figure}
    \centering
    \includegraphics[width=1\textwidth]{Feigenbaum Approx Graphs/feigenbaum_scaled.png}
    \caption{Rescaled graphs with scaling factor $\mu=2.552571093277968$, corresponding to the logistic map which shows the convergence to the limiting function as we iterate $r,k$ that aligns Feigenbaum's numerics.}
    \label{fig:rescaled}
\end{figure}
The blue lines again show us Feigenbaum's numerically solved solution \eqref{eq:feigenbaum}. 
We can now see that due to $\mu$, our rescaled $g(x)$ now shifts up towards Feigenbaum's solution and attains the desired shape showing us that we have successfully rescaled our original solution to the universal one presented by Feigenbaum. 
Thus we have demonstrated the convergence described by Feigenbaum towards his numerical solution as well as the existence of a limiting function $g(x)$ as we increase both $r$ and $k$ simultaneously. 
\end{exmp}

\section{Numerical Computation of Eigenfunction of $T$}

To make the above section rigorous, and to prove the existence of an eigenfunction of the operator $T$, is non-trivial and requires a hundred-page proof \cite{lyubich1999feigenbaum}, let alone finding such a function, and is clearly out of the scope of an undergraduate mathematical report like this one. 
In this section, we will attempt to compute the first few terms of the power series of this eigenfunction.

Assuming $g(0) = 1$ and that $g$ is an even function, (the validity of such assumption is justified by the fact that a successful result was found based on it), it can be written in the form of Taylor expansion 
$$
g = 1 + b_1 x^2 + b_2x^4 + \cdots 
$$
Defining the $n$th partial sum of this power series as $g = 1 + b_1x^2 + \cdots + b_{n-1}x^{2n-2}$, and applying $T$ to $g_n$ will produce another $(2n-2)^2$-degree polynomial whose coefficients depend on $b_1, \cdots, b_{n-1}$, and $\alpha$.
Ignoring the terms of degree higher than $2n-2$, and equating the coefficients of the result with that of $g$ of the same degree will give $n$ functions for $n$ variables, whose solution can be solved numerically. 

We will demonstrate an example for $g_1 = 1 + b_1 x^2$. 

Note 
$$
g_1\left(-\frac{x}{\alpha}\right) = 1 + b_1 \frac{x^2}{\alpha ^2}
$$

and, by plugging in,

\begin{align*}
-\alpha g_1\left(g\left(-\frac{x}{\alpha}\right)\right) 
&= -\alpha\left( 1 + b_1\left(1+b_1 \frac{x^2}{\alpha^2}\right)^2\right) \\
&= -\alpha\left(1 + b_1\left(1 + b_1^2\frac{x^4}{\alpha^4} + x^2 \frac{2b_1}{\alpha^2}\right)\right) \\
&= -\alpha (1 + b_1)  - \frac{2b_1^2}{\alpha} x^2 - \frac{b_1^3}{\alpha^3}x^4
\end{align*}

thereby comes the equation 

$$
\begin{cases}
    1 &= -\alpha(1+b_1) \\
    b_1 &= -\frac{2b_1^2}{\alpha} 
\end{cases}
$$

The solution is $\alpha = 1 + \sqrt{3} \approx 2.732$ and $b_1 = - \frac{\sqrt{3}}{2} - \frac{1}{2} \approx -1.366$.

Calculating any more terms by hand would be an extremely tedious process; luckily, we can resort to computer algebra systems for symbolic manipulation (we used Sympy) and BLAS library (fsolve from Scipy) for solving equations of $n$ variables. 
The approximation for the fifteen terms is shown in Table \ref{tb:b_i table}.
The python code for producing these results is included in Appendix \ref{fsolve}.

By this calculation, and using Equation \eqref{eq:alpha}, an estimation of $\alpha$ is obtained to be $2.5029078754977733823$. We are satisfied with this result as it follows closely to Feigenbaum's own result, implying that our method was successful. Being able to find this constant within a chaotic dynamical system is truly fascinating and a true achievement of Feigenbaum. 

\begin{table}
\centering
\begin{tabular}{|c|c|}
\hline
$n$ & $b_n$ \\
\hline
1 & \( -1.527632997 \times 10^{0} \) \\
2 & \( 1.048151948 \times 10^{-1} \) \\
3 & \( 2.670567058 \times 10^{-2} \) \\
4 & \( -3.527409767 \times 10^{-3} \) \\
5 & \( 8.160111810 \times 10^{-5} \) \\
6 & \( 2.528490972 \times 10^{-5} \) \\
7 & \( -2.556150238 \times 10^{-6} \) \\
8 & \( -9.664817874 \times 10^{-8} \) \\
9 & \( 2.828824961 \times 10^{-8} \) \\
10 & \( -3.352250769 \times 10^{-10} \) \\
11 & \( -2.716117768 \times 10^{-10} \) \\
12 & \( 1.171671302 \times 10^{-11} \) \\
13 & \( 7.447301044 \times 10^{-12} \) \\
14 & \( -2.868216370 \times 10^{-12} \) \\
15 & \( 9.052335129 \times 10^{-13} \) \\
\hline
\end{tabular}
\caption{The values of $b_n$ are the coefficients of the Taylor expression of $g$ such that $g = 1 + \sum_{i=1}^n b_i x^{2i}$, where $g$ is the solution to $Tg = g$.
}
\label{tb:b_i table}
\end{table}




% \addcontentsline{toc}{chapter}{Conclusion}
\chapter*{Conclusion}
\addcontentsline{toc}{chapter}{Conclusion}
In this report, we identified chaotic dynamical systems, showing the consequences of the Lyapunov exponent on time-dependent functions. We proceeded to investigate quantifiable properties of these systems, motivated by an interest in the behaviour of such complicated dynamics. Throughout, we built theory and provided key examples of each topic in order to introduce and discover these properties. As well as this, we built our own computer code to assist with visualisations and numerical approximations, steps which, throughout history, have proven to be highly important in the study of dynamical systems. 

The crescendo of our report was in approximating the Feigenbaum constants; we were able to surpass Feigenbaum's numerical computations with the use of modern computer technologies and derive an approximation of the constants to a high degree of accuracy.
There are, of course, room for improvement of numerical accuracy, which will necessarily require truly ingenious algorithms and mastery of programming techniques. 
It would be suitable for an undergraduate project for future generations.

The field of chaotic dynamical systems is vast and spectacular, and this report has but touched the very surface. 
There are countless questions to be asked. 
What are the analytical properties of the Feigenbaum constants? 
Are they rational, irrational, or transcendental? 
Do they have a closed form or any relationship with other fields of mathematics?
This report has only investigated the Feigenbaum constants corresponding to the periodic doubling bifurcations of discrete dynamical systems of one parameter. 
Periodic bifurcation of triple and higher multitudes, and that of more parameters and continuous systems, are known to exhibit similar behaviours and can be described by other universal constants. 
How to generalise the methods in this report to these more complicated systems? What would these constants be, and how do they relate to each other?
They will constitute interesting topics for future research well beyond the scope of an undergraduate project.


% \bibliographystyle{...} 
\printbibliography
\addcontentsline{toc}{chapter}{Bibliography}

\begin{appendices}
	\chapter{Numerical Calculations for Point of Super-stability}
	\input{numerics_cpp.tex}
    \chapter{Solving Coefficients of Taylor Series in Python}
	\input{fsolve_for_feigenbaum_coef.tex}
    \chapter{Numerics for Lyapunov Exponents in Python}
    \begin{lstlisting}[style=python]
def logistic(r, x):
    return 4*r * x * (1 - x)

r_high = 1
r_low = 0
# Parameters
r_values = np.linspace(r_low, r_high, 10000)
iterations = 1000
last = 100

#Bifurcation diagram
fig1, ax1 = plt.subplots(figsize=(10, 5))
x = 1e-5 * np.ones_like(r_values)

for i in range(iterations):
    x = logistic(r_values, x)
    if i >= (iterations - last):
        ax1.plot(r_values, x, ',k', alpha=.25, c = 'tab:blue')

ax1.set_xlabel('r')
ax1.set_ylabel('x')

ax1.grid(True)
fig1.tight_layout()
plt.show()

# Lyapunov exponent plot
fig2, ax2 = plt.subplots(figsize=(10, 5))
lyapunov = np.zeros_like(r_values)

x = 1e-5 * np.ones_like(r_values)  # Reset x for Lyapunov calculation
for i in range(iterations):
    x = logistic(r_values, x)
    lyapunov += np.log(abs(4*r_values - 8 * r_values * x))

ax2.plot(r_values, lyapunov / iterations, 'k', alpha=.9, c="tab:blue")
ax2.set_xlabel('r')
ax2.set_ylabel('Lyapunov exponent')

# Set x-ticks
# ax2.set_xticks(np.arange(2.5, 4.1, 0.5))
# ax2.set_xticklabels([f'{tick:.1f}' for tick in np.arange(2.5, 4.1, 0.5)])

ax2.grid(True)
fig2.tight_layout()
plt.show()
\end{lstlisting}

    \chapter{Box-Counting Method in Python}
    \begin{lstlisting}[style=python]
import numpy as np
import matplotlib.pyplot as plt
import matplotlib.patches as patches

# Define the Henon map function
def henon_attractor(a=1.4, b=0.3, n_points=10000000, burn=10000):
    """
    Generates a number of points from the Henon attractor, with standard parameters, discarding some early iterations.
    
    """
    x, y = 0, 0  # Initial conditions
    x_vals, y_vals = [], []
    total_points = n_points + burn
    for i in range(total_points):
        x, y = 1 - a * x**2 + y, b * x
        if i >= burn:
            x_vals.append(x)
            y_vals.append(y)
    return np.array(x_vals), np.array(y_vals)

=
def draw_blue_boxes(x_vals, y_vals, box_size, ax, xlim, ylim):
    """
    Create a grid and find boxes that contain at least one point from the attractor, colour these boxes blue and discard the rest.
    """
    # Create grid edges using the fixed axis limits, and rescale the y edges so that the boxes appear sqaure for presentation (this does not affect our results)
    x_edges = np.arange(xlim[0], xlim[1] + box_size, box_size)
    y_edges = np.arange(ylim[0], ylim[1] + box_size/3.5, box_size/3.5)
    
    # Compute a 2D histogram to determine occupancy of each grid cell
    H, x_edges, y_edges = np.histogram2d(x_vals, y_vals, bins=[x_edges, y_edges])
    
    # For each grid cell that contains at least one point, draw a blue box
    for i in range(H.shape[0]):
        for j in range(H.shape[1]):
            if H[i, j] > 0:
                lower_left = (x_edges[i], y_edges[j])
                rect = patches.Rectangle(lower_left, box_size, box_size/3.5,
                                         fill=False, edgecolor='blue', lw=0.5)
                ax.add_patch(rect)

# Count the number of occupied boxes for some box size, given by the length of one 
def count_boxes(x_vals, y_vals, box_size, xlim, ylim):
    x_edges = np.arange(xlim[0], xlim[1] + box_size, box_size)
    y_edges = np.arange(ylim[0], ylim[1] + box_size/3.5, box_size/3.5)
    H, _, _ = np.histogram2d(x_vals, y_vals, bins=[x_edges, y_edges])
    return np.sum(H > 0)

# Compute box-counting dimension and plot log-log graph
def compute_box_counting_dimension(x_vals, y_vals, xlim, ylim):
    box_sizes = np.logspace(-1, 0, num=10)
    counts = [count_boxes(x_vals, y_vals, bs, xlim, ylim) for bs in box_sizes]
    log_inv_bs = np.log(1 / np.array(box_sizes))
    log_counts = np.log(counts)
    slope, intercept = np.polyfit(log_inv_bs, log_counts, 1)
    
    ax.tick_params(axis='both', which='major', labelsize=12)
    plt.figure(figsize=(8, 8))
    plt.scatter(log_inv_bs, log_counts, color='blue', label='Data')
    plt.plot(log_inv_bs, slope * log_inv_bs + intercept, 'r--', label=f'Fit: D={slope:.3f}')
    plt.xlabel('log(1/box size)', size = 20)
    plt.ylabel('log(box count)', size = 20)
    plt.legend(fontsize = 16)
    plt.show()
    
    return slope

# Main script
# Generate the Henon attractor points (with burn-in)
x_vals, y_vals = henon_attractor(n_points=10000000, burn=10000)

xlim = (np.min(x_vals), np.max(x_vals))
ylim = (-0.4, 0.4)

# Create three subplots with three different box sizes (from large to small)
fig, axs = plt.subplots(1, 3, figsize=(18, 7))

# Define three representative box sizes (you can adjust these values)
box_sizes = [0.1, 0.05, 0.02]
titles = [f"Box size = {bs:.2f}" for bs in box_sizes]

for ax, bs, title in zip(axs, box_sizes, titles):
    ax.plot(x_vals, y_vals, ',k', alpha=0.5)
    ax.set_xlim(xlim)
    ax.set_ylim(ylim)
    ax.set_xlabel('x', size = 20)
    ax.set_ylabel('y', size = 20)
    ax.set_title(title, size = 20)
    ax.tick_params(axis='both', which='major', labelsize=12)
    # Draw the blue boxes over the attractor
    draw_blue_boxes(x_vals, y_vals, bs, ax, xlim, ylim)

plt.tight_layout(rect=[0, 0.03, 1, 0.95])
plt.show()
\end{lstlisting}\label{boxalg}
\end{appendices}
\end{document}

