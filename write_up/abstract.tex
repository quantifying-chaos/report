\begin{abstract}
Chaos is a universal phenomenon observed in every aspect of the physical world. Examples include weather systems, celestial movements, demographic evolutions, and economic activities. 
The descriptions thereof often comprise of the adjectives `unpredictable,' `disordered,' `entropic,' `confusing,' `subject to chances,' and, in summary, `chaotic.'
It may be of surprise that much more can be said to even the simplest examples.
In this report we will demonstrate there are rich, universal, and quantitative mathematical properties shared by all of these chaotic systems.
This report discovers the order within the chaos.

We begin by introducing examples of chaos and their defining characteristics. 
The intuitive descriptions of chaos, as will be shown, can be rigorously described by Lyapunov exponents.
Then, we move on to investigate a class of objects that are a natural result of chaos. These objects appear odd, and will require the quantifiable property of non-integer dimensions.
After this, the iterative logistic map as a model of population is studied in detail. The map exhibits an interesting behaviour that results in chaos that is also observed among a large class of similar systems. 
After developing some mathematically rigorous tools to study chaos, the final step is the derivation of Feigenbaum's universal constants, which are found to be shared across the wide class of systems we previously investigate.


\end{abstract}