In this report, we identified chaotic dynamical systems, showing the consequences of the Lyapunov exponent on time-dependent functions. We proceeded to investigate quantifiable properties of these systems, motivated by an interest in the behaviour of such complicated dynamics. Throughout, we built theory and provided key examples of each topic in order to introduce and discover these properties. As well as this, we built our own computer code to assist with visualisations and numerical approximations, steps which, throughout history, have proven to be highly important in the study of dynamical systems. 

The crescendo of our report was in approximating the Feigenbaum constants; we were able to surpass Feigenbaum's numerical computations with the use of modern computer technologies and derive an approximation of the constants to a high degree of accuracy.
There are, of course, room for improvement of numerical accuracy, which will necessarily require truly ingenious algorithms and mastery of programming techniques. 
It would be suitable for an undergraduate project for future generations.

The field of chaotic dynamical systems is vast and spectacular, and this report has but touched the very surface. 
There are countless questions to be asked. 
What are the analytical properties of the Feigenbaum constants? 
Are they rational, irrational, or transcendental? 
Do they have a closed form or any relationship with other fields of mathematics?
This report has only investigated the Feigenbaum constants corresponding to the periodic doubling bifurcations of discrete dynamical systems of one parameter. 
Periodic bifurcation of triple and higher multitudes, and that of more parameters and continuous systems, are known to exhibit similar behaviours and can be described by other universal constants. 
How to generalise the methods in this report to these more complicated systems? What would these constants be, and how do they relate to each other?
They will constitute interesting topics for future research well beyond the scope of an undergraduate project.