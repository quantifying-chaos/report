In this report, we identified chaotic dynamical systems, showing the consequences of the Lyapunov exponent on time-dependent functions. We proceeded to investigate quantifiable properties of these systems, motivated by an interest in the behaviour of such complicated dynamics. Throughout, we built theory and provided key examples of each topic in order to introduce and discover these properties. As well as this, we built our own computer code to assist with visualisations and numerical approximations, steps which, throughout history, have proven to be highly important in the study of dynamical systems. 

The crescendo of our report was in approximating the Feigenbaum constants; we were able to surpass Feigenbaum's numerical computations with the use of modern computer code and derive an approximation of the constants to a high degree of accuracy of the true values. Given the opportunity, the next step in our report would be to rigorously prove the existence and uniqueness of these constants, without involving numerical approximations. This task further builds on the theory we have introduced but exceeds the scope of our study.